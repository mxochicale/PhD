%% bare_jrnl_compsoc.tex
%% V1.4a
%% 2014/09/17
%% by Michael Shell
%% See:
%% http://www.michaelshell.org/
%% for current contact information.
%%
%% This is a skeleton file demonstrating the use of IEEEtran.cls
%% (requires IEEEtran.cls version 1.8a or later) with an IEEE
%% Computer Society journal paper.
%%
%% Support sites:
%% http://www.michaelshell.org/tex/ieeetran/
%% http://www.ctan.org/tex-archive/macros/latex/contrib/IEEEtran/
%% and
%% http://www.ieee.org/

%%*************************************************************************
%% Legal Notice:
%% This code is offered as-is without any warranty either expressed or
%% implied; without even the implied warranty of MERCHANTABILITY or
%% FITNESS FOR A PARTICULAR PURPOSE! 
%% User assumes all risk.
%% In no event shall IEEE or any contributor to this code be liable for
%% any damages or losses, including, but not limited to, incidental,
%% consequential, or any other damages, resulting from the use or misuse
%% of any information contained here.
%%
%% All comments are the opinions of their respective authors and are not
%% necessarily endorsed by the IEEE.
%%
%% This work is distributed under the LaTeX Project Public License (LPPL)
%% ( http://www.latex-project.org/ ) version 1.3, and may be freely used,
%% distributed and modified. A copy of the LPPL, version 1.3, is included
%% in the base LaTeX documentation of all distributions of LaTeX released
%% 2003/12/01 or later.
%% Retain all contribution notices and credits.
%% ** Modified files should be clearly indicated as such, including  **
%% ** renaming them and changing author support contact information. **
%%
%% File list of work: IEEEtran.cls, IEEEtran_HOWTO.pdf, bare_adv.tex,
%%                    bare_conf.tex, bare_jrnl.tex, bare_conf_compsoc.tex,
%%                    bare_jrnl_compsoc.tex, bare_jrnl_transmag.tex
%%*************************************************************************


% *** Authors should verify (and, if needed, correct) their LaTeX system  ***
% *** with the testflow diagnostic prior to trusting their LaTeX platform ***
% *** with production work. IEEE's font choices and paper sizes can       ***
% *** trigger bugs that do not appear when using other class files.       ***                          ***
% The testflow support page is at:
% http://www.michaelshell.org/tex/testflow/


\documentclass[10pt,journal,compsoc]{IEEEtran}
%
% If IEEEtran.cls has not been installed into the LaTeX system files,
% manually specify the path to it like:
% \documentclass[10pt,journal,compsoc]{../sty/IEEEtran}





% Some very useful LaTeX packages include:
% (uncomment the ones you want to load)


% *** MISC UTILITY PACKAGES ***
%
%\usepackage{ifpdf}
% Heiko Oberdiek's ifpdf.sty is very useful if you need conditional
% compilation based on whether the output is pdf or dvi.
% usage:
% \ifpdf
%   % pdf code
% \else
%   % dvi code
% \fi
% The latest version of ifpdf.sty can be obtained from:
% http://www.ctan.org/tex-archive/macros/latex/contrib/oberdiek/
% Also, note that IEEEtran.cls V1.7 and later provides a builtin
% \ifCLASSINFOpdf conditional that works the same way.
% When switching from latex to pdflatex and vice-versa, the compiler may
% have to be run twice to clear warning/error messages.






% *** CITATION PACKAGES ***
%
\ifCLASSOPTIONcompsoc
  % IEEE Computer Society needs nocompress option
  % requires cite.sty v4.0 or later (November 2003)
  \usepackage[nocompress]{cite}
\else
  % normal IEEE
  \usepackage{cite}
\fi
% cite.sty was written by Donald Arseneau
% V1.6 and later of IEEEtran pre-defines the format of the cite.sty package
% \cite{} output to follow that of IEEE. Loading the cite package will
% result in citation numbers being automatically sorted and properly
% "compressed/ranged". e.g., [1], [9], [2], [7], [5], [6] without using
% cite.sty will become [1], [2], [5]--[7], [9] using cite.sty. cite.sty's
% \cite will automatically add leading space, if needed. Use cite.sty's
% noadjust option (cite.sty V3.8 and later) if you want to turn this off
% such as if a citation ever needs to be enclosed in parenthesis.
% cite.sty is already installed on most LaTeX systems. Be sure and use
% version 5.0 (2009-03-20) and later if using hyperref.sty.
% The latest version can be obtained at:
% http://www.ctan.org/tex-archive/macros/latex/contrib/cite/
% The documentation is contained in the cite.sty file itself.
%
% Note that some packages require special options to format as the Computer
% Society requires. In particular, Computer Society  papers do not use
% compressed citation ranges as is done in typical IEEE papers
% (e.g., [1]-[4]). Instead, they list every citation separately in order
% (e.g., [1], [2], [3], [4]). To get the latter we need to load the cite
% package with the nocompress option which is supported by cite.sty v4.0
% and later. Note also the use of a CLASSOPTION conditional provided by
% IEEEtran.cls V1.7 and later.





% *** GRAPHICS RELATED PACKAGES ***
%
\ifCLASSINFOpdf
  \usepackage[pdftex]{graphicx}
  % declare the path(s) where your graphic files are
  % \graphicspath{{../pdf/}{../jpeg/}}
  % and their extensions so you won't have to specify these with
  % every instance of \includegraphics
  % \DeclareGraphicsExtensions{.pdf,.jpeg,.png}
\else
  % or other class option (dvipsone, dvipdf, if not using dvips). graphicx
  % will default to the driver specified in the system graphics.cfg if no
  % driver is specified.
  % \usepackage[dvips]{graphicx}
  % declare the path(s) where your graphic files are
  % \graphicspath{{../eps/}}
  % and their extensions so you won't have to specify these with
  % every instance of \includegraphics
  % \DeclareGraphicsExtensions{.eps}
\fi
% graphicx was written by David Carlisle and Sebastian Rahtz. It is
% required if you want graphics, photos, etc. graphicx.sty is already
% installed on most LaTeX systems. The latest version and documentation
% can be obtained at: 
% http://www.ctan.org/tex-archive/macros/latex/required/graphics/
% Another good source of documentation is "Using Imported Graphics in
% LaTeX2e" by Keith Reckdahl which can be found at:
% http://www.ctan.org/tex-archive/info/epslatex/
%
% latex, and pdflatex in dvi mode, support graphics in encapsulated
% postscript (.eps) format. pdflatex in pdf mode supports graphics
% in .pdf, .jpeg, .png and .mps (metapost) formats. Users should ensure
% that all non-photo figures use a vector format (.eps, .pdf, .mps) and
% not a bitmapped formats (.jpeg, .png). IEEE frowns on bitmapped formats
% which can result in "jaggedy"/blurry rendering of lines and letters as
% well as large increases in file sizes.
%
% You can find documentation about the pdfTeX application at:
% http://www.tug.org/applications/pdftex






% *** MATH PACKAGES ***
%
\usepackage[cmex10]{amsmath}
% A popular package from the American Mathematical Society that provides
% many useful and powerful commands for dealing with mathematics. If using
% it, be sure to load this package with the cmex10 option to ensure that
% only type 1 fonts will utilized at all point sizes. Without this option,
% it is possible that some math symbols, particularly those within
% footnotes, will be rendered in bitmap form which will result in a
% document that can not be IEEE Xplore compliant!
%
% Also, note that the amsmath package sets \interdisplaylinepenalty to 10000
% thus preventing page breaks from occurring within multiline equations. Use:
%\interdisplaylinepenalty=2500
% after loading amsmath to restore such page breaks as IEEEtran.cls normally
% does. amsmath.sty is already installed on most LaTeX systems. The latest
% version and documentation can be obtained at:
% http://www.ctan.org/tex-archive/macros/latex/required/amslatex/math/


\usepackage{amsfonts} % to use $\mathbb{Z}$


% *** SPECIALIZED LIST PACKAGES ***

\usepackage{algorithm}
\usepackage{algorithmic}
% algorithmic.sty was written by Peter Williams and Rogerio Brito.
% This package provides an algorithmic environment fo describing algorithms.
% You can use the algorithmic environment in-text or within a figure
% environment to provide for a floating algorithm. Do NOT use the algorithm
% floating environment provided by algorithm.sty (by the same authors) or
% algorithm2e.sty (by Christophe Fiorio) as IEEE does not use dedicated
% algorithm float types and packages that provide these will not provide
% correct IEEE style captions. The latest version and documentation of
% algorithmic.sty can be obtained at:
% http://www.ctan.org/tex-archive/macros/latex/contrib/algorithms/
% There is also a support site at:
% http://algorithms.berlios.de/index.html
% Also of interest may be the (relatively newer and more customizable)
% algorithmicx.sty package by Szasz Janos:
% http://www.ctan.org/tex-archive/macros/latex/contrib/algorithmicx/




% *** ALIGNMENT PACKAGES ***
%
%\usepackage{array}
% Frank Mittelbach's and David Carlisle's array.sty patches and improves
% the standard LaTeX2e array and tabular environments to provide better
% appearance and additional user controls. As the default LaTeX2e table
% generation code is lacking to the point of almost being broken with
% respect to the quality of the end results, all users are strongly
% advised to use an enhanced (at the very least that provided by array.sty)
% set of table tools. array.sty is already installed on most systems. The
% latest version and documentation can be obtained at:
% http://www.ctan.org/tex-archive/macros/latex/required/tools/


% IEEEtran contains the IEEEeqnarray family of commands that can be used to
% generate multiline equations as well as matrices, tables, etc., of high
% quality.




% *** SUBFIGURE PACKAGES ***
%\ifCLASSOPTIONcompsoc
%  \usepackage[caption=false,font=footnotesize,labelfont=sf,textfont=sf]{subfig}
%\else
%  \usepackage[caption=false,font=footnotesize]{subfig}
%\fi
% subfig.sty, written by Steven Douglas Cochran, is the modern replacement
% for subfigure.sty, the latter of which is no longer maintained and is
% incompatible with some LaTeX packages including fixltx2e. However,
% subfig.sty requires and automatically loads Axel Sommerfeldt's caption.sty
% which will override IEEEtran.cls' handling of captions and this will result
% in non-IEEE style figure/table captions. To prevent this problem, be sure
% and invoke subfig.sty's "caption=false" package option (available since
% subfig.sty version 1.3, 2005/06/28) as this is will preserve IEEEtran.cls
% handling of captions.
% Note that the Computer Society format requires a sans serif font rather
% than the serif font used in traditional IEEE formatting and thus the need
% to invoke different subfig.sty package options depending on whether
% compsoc mode has been enabled.
%
% The latest version and documentation of subfig.sty can be obtained at:
% http://www.ctan.org/tex-archive/macros/latex/contrib/subfig/




% *** FLOAT PACKAGES ***
%
%\usepackage{fixltx2e}
% fixltx2e, the successor to the earlier fix2col.sty, was written by
% Frank Mittelbach and David Carlisle. This package corrects a few problems
% in the LaTeX2e kernel, the most notable of which is that in current
% LaTeX2e releases, the ordering of single and double column floats is not
% guaranteed to be preserved. Thus, an unpatched LaTeX2e can allow a
% single column figure to be placed prior to an earlier double column
% figure. The latest version and documentation can be found at:
% http://www.ctan.org/tex-archive/macros/latex/base/


%\usepackage{stfloats}
% stfloats.sty was written by Sigitas Tolusis. This package gives LaTeX2e
% the ability to do double column floats at the bottom of the page as well
% as the top. (e.g., "\begin{figure*}[!b]" is not normally possible in
% LaTeX2e). It also provides a command:
%\fnbelowfloat
% to enable the placement of footnotes below bottom floats (the standard
% LaTeX2e kernel puts them above bottom floats). This is an invasive package
% which rewrites many portions of the LaTeX2e float routines. It may not work
% with other packages that modify the LaTeX2e float routines. The latest
% version and documentation can be obtained at:
% http://www.ctan.org/tex-archive/macros/latex/contrib/sttools/
% Do not use the stfloats baselinefloat ability as IEEE does not allow
% \baselineskip to stretch. Authors submitting work to the IEEE should note
% that IEEE rarely uses double column equations and that authors should try
% to avoid such use. Do not be tempted to use the cuted.sty or midfloat.sty
% packages (also by Sigitas Tolusis) as IEEE does not format its papers in
% such ways.
% Do not attempt to use stfloats with fixltx2e as they are incompatible.
% Instead, use Morten Hogholm'a dblfloatfix which combines the features
% of both fixltx2e and stfloats:
%
% \usepackage{dblfloatfix}
% The latest version can be found at:
% http://www.ctan.org/tex-archive/macros/latex/contrib/dblfloatfix/




%\ifCLASSOPTIONcaptionsoff
%  \usepackage[nomarkers]{endfloat}
% \let\MYoriglatexcaption\caption
% \renewcommand{\caption}[2][\relax]{\MYoriglatexcaption[#2]{#2}}
%\fi
% endfloat.sty was written by James Darrell McCauley, Jeff Goldberg and 
% Axel Sommerfeldt. This package may be useful when used in conjunction with 
% IEEEtran.cls'  captionsoff option. Some IEEE journals/societies require that
% submissions have lists of figures/tables at the end of the paper and that
% figures/tables without any captions are placed on a page by themselves at
% the end of the document. If needed, the draftcls IEEEtran class option or
% \CLASSINPUTbaselinestretch interface can be used to increase the line
% spacing as well. Be sure and use the nomarkers option of endfloat to
% prevent endfloat from "marking" where the figures would have been placed
% in the text. The two hack lines of code above are a slight modification of
% that suggested by in the endfloat docs (section 8.4.1) to ensure that
% the full captions always appear in the list of figures/tables - even if
% the user used the short optional argument of \caption[]{}.
% IEEE papers do not typically make use of \caption[]'s optional argument,
% so this should not be an issue. A similar trick can be used to disable
% captions of packages such as subfig.sty that lack options to turn off
% the subcaptions:
% For subfig.sty:
% \let\MYorigsubfloat\subfloat
% \renewcommand{\subfloat}[2][\relax]{\MYorigsubfloat[]{#2}}
% However, the above trick will not work if both optional arguments of
% the \subfloat command are used. Furthermore, there needs to be a
% description of each subfigure *somewhere* and endfloat does not add
% subfigure captions to its list of figures. Thus, the best approach is to
% avoid the use of subfigure captions (many IEEE journals avoid them anyway)
% and instead reference/explain all the subfigures within the main caption.
% The latest version of endfloat.sty and its documentation can obtained at:
% http://www.ctan.org/tex-archive/macros/latex/contrib/endfloat/
%
% The IEEEtran \ifCLASSOPTIONcaptionsoff conditional can also be used
% later in the document, say, to conditionally put the References on a 
% page by themselves.




% *** PDF, URL AND HYPERLINK PACKAGES ***
%
%\usepackage{url}
% url.sty was written by Donald Arseneau. It provides better support for
% handling and breaking URLs. url.sty is already installed on most LaTeX
% systems. The latest version and documentation can be obtained at:
% http://www.ctan.org/tex-archive/macros/latex/contrib/url/
% Basically, \url{my_url_here}.





% *** Do not adjust lengths that control margins, column widths, etc. ***
% *** Do not use packages that alter fonts (such as pslatex).         ***
% There should be no need to do such things with IEEEtran.cls V1.6 and later.
% (Unless specifically asked to do so by the journal or conference you plan
% to submit to, of course. )


% correct bad hyphenation here
\hyphenation{op-tical net-works semi-conduc-tor}


\begin{document}
%
% paper title
% Titles are generally capitalized except for words such as a, an, and, as,
% at, but, by, for, in, nor, of, on, or, the, to and up, which are usually
% not capitalized unless they are the first or last word of the title.
% Linebreaks \\ can be used within to get better formatting as desired.
% Do not put math or special symbols in the title.
% \title{Bare Demo of IEEEtran.cls\\ for Computer Society Journals}
\title{Automatic Non-linear Analysis of the Variability of Human Activities}
%
%
% author names and IEEE memberships
% note positions of commas and nonbreaking spaces ( ~ ) LaTeX will not break
% a structure at a ~ so this keeps an author's name from being broken across
% two lines.
% use \thanks{} to gain access to the first footnote area
% a separate \thanks must be used for each paragraph as LaTeX2e's \thanks
% was not built to handle multiple paragraphs
%
%
%\IEEEcompsocitemizethanks is a special \thanks that produces the bulleted
% lists the Computer Society journals use for "first footnote" author
% affiliations. Use \IEEEcompsocthanksitem which works much like \item
% for each affiliation group. When not in compsoc mode,
% \IEEEcompsocitemizethanks becomes like \thanks and
% \IEEEcompsocthanksitem becomes a line break with idention. This
% facilitates dual compilation, although admittedly the differences in the
% desired content of \author between the different types of papers makes a
% one-size-fits-all approach a daunting prospect. For instance, compsoc 
% journal papers have the author affiliations above the "Manuscript
% received ..."  text while in non-compsoc journals this is reversed. Sigh.

\author{Miguel P\'erez-Xochicale
% 	Michael~Shell,~\IEEEmembership{Member,~IEEE,}
%         John~Doe,~\IEEEmembership{Fellow,~OSA,}
%         and~Jane~Doe,~\IEEEmembership{Life~Fellow,~IEEE}% <-this % stops a space
% \IEEEcompsocitemizethanks{\IEEEcompsocthanksitem M. Shell is with the Department
% of Electrical and Computer Engineering, Georgia Institute of Technology, Atlanta,
% GA, 30332.\protect\\
% % note need leading \protect in front of \\ to get a newline within \thanks as
% % \\ is fragile and will error, could use \hfil\break instead.
% E-mail: see http://www.michaelshell.org/contact.html
% \IEEEcompsocthanksitem J. Doe and J. Doe are with Anonymous University.}% <-this 
% stops an unwanted space
% \thanks{Manuscript received April 19, 2005; revised September 17, 2014.}
}

% note the % following the last \IEEEmembership and also \thanks - 
% these prevent an unwanted space from occurring between the last author name
% and the end of the author line. i.e., if you had this:
% 
% \author{....lastname \thanks{...} \thanks{...} }
%                     ^------------^------------^----Do not want these spaces!
%
% a space would be appended to the last name and could cause every name on that
% line to be shifted left slightly. This is one of those "LaTeX things". For
% instance, "\textbf{A} \textbf{B}" will typeset as "A B" not "AB". To get
% "AB" then you have to do: "\textbf{A}\textbf{B}"
% \thanks is no different in this regard, so shield the last } of each \thanks
% that ends a line with a % and do not let a space in before the next \thanks.
% Spaces after \IEEEmembership other than the last one are OK (and needed) as
% you are supposed to have spaces between the names. For what it is worth,
% this is a minor point as most people would not even notice if the said evil
% space somehow managed to creep in.



% The paper headers
\markboth{Ninth Month Report, August~2015}%
{Shell \MakeLowercase{\textit{et al.}}: Bare Demo of IEEEtran.cls for Computer 
Society Journals}
% The only time the second header will appear is for the odd numbered pages
% after the title page when using the twoside option.
% 
% *** Note that you probably will NOT want to include the author's ***
% *** name in the headers of peer review papers.                   ***
% You can use \ifCLASSOPTIONpeerreview for conditional compilation here if
% you desire.



% The publisher's ID mark at the bottom of the page is less important with
% Computer Society journal papers as those publications place the marks
% outside of the main text columns and, therefore, unlike regular IEEE
% journals, the available text space is not reduced by their presence.
% If you want to put a publisher's ID mark on the page you can do it like
% this:
%\IEEEpubid{0000--0000/00\$00.00~\copyright~2014 IEEE}
% or like this to get the Computer Society new two part style.
%\IEEEpubid{\makebox[\columnwidth]{\hfill 0000--0000/00/\$00.00~\copyright~2014 IEEE}%
%\hspace{\columnsep}\makebox[\columnwidth]{Published by the IEEE Computer Society\hfill}}
% Remember, if you use this you must call \IEEEpubidadjcol in the second
% column for its text to clear the IEEEpubid mark (Computer Society jorunal
% papers don't need this extra clearance.)



% use for special paper notices
%\IEEEspecialpapernotice{(Invited Paper)}



% for Computer Society papers, we must declare the abstract and index terms
% PRIOR to the title within the \IEEEtitleabstractindextext IEEEtran
% command as these need to go into the title area created by \maketitle.
% As a general rule, do not put math, special symbols or citations
% in the abstract or keywords.
\IEEEtitleabstractindextext{%
\begin{abstract}
% The abstract goes here.
\end{abstract}

% Note that keywords are not normally used for peerreview papers.
\begin{IEEEkeywords}
Activity Recognition; On-Body Inertial Sensors
% ; Motor Skill Assessment; Human-Robot Interaction
\end{IEEEkeywords}}


% make the title area
\maketitle


% To allow for easy dual compilation without having to reenter the
% abstract/keywords data, the \IEEEtitleabstractindextext text will
% not be used in maketitle, but will appear (i.e., to be "transported")
% here as \IEEEdisplaynontitleabstractindextext when the compsoc 
% or transmag modes are not selected <OR> if conference mode is selected 
% - because all conference papers position the abstract like regular
% papers do.
\IEEEdisplaynontitleabstractindextext
% \IEEEdisplaynontitleabstractindextext has no effect when using
% compsoc or transmag under a non-conference mode.



% For peer review papers, you can put extra information on the cover
% page as needed:
% \ifCLASSOPTIONpeerreview
% \begin{center} \bfseries EDICS Category: 3-BBND \end{center}
% \fi
%
% For peerreview papers, this IEEEtran command inserts a page break and
% creates the second title. It will be ignored for other modes.
\IEEEpeerreviewmaketitle



\IEEEraisesectionheading{\section{Introduction}\label{sec:introduction}}
% Computer Society journal (but not conference!) papers do something unusual
% with the very first section heading (almost always called "Introduction").
% They place it ABOVE the main text! IEEEtran.cls does not automatically do
% this for you, but you can achieve this effect with the provided
% \IEEEraisesectionheading{} command. Note the need to keep any \label that
% is to refer to the section immediately after \section in the above as
% \IEEEraisesectionheading puts \section within a raised box.

% The very first letter is a 2 line initial drop letter followed
% by the rest of the first word in caps (small caps for compsoc).
% 
% form to use if the first word consists of a single letter:
% \IEEEPARstart{A}{demo} file is ....
% 
% form to use if you need the single drop letter followed by
% normal text (unknown if ever used by IEEE):
% \IEEEPARstart{A}{}demo file is ....
% 
% Some journals put the first two words in caps:
% \IEEEPARstart{T}{his demo} file is ....
% 
% Here we have the typical use of a "T" for an initial drop letter
% and "HIS" in caps to complete the first word.


\IEEEPARstart{H}{uman} Activity Recognition (HAR) using body-worn sensors 
has received interest during the last 20 years \cite{bulling2014, Lara2013}. 
This is due to three factors: (i) technology advances in sensors, 
(ii) longer battery lifetimes and  (iii) different application-oriented scenarios.
In contrast to speech recognition and computer vision frameworks, HAR offers different 
challenges based on the complexity and diversity of human activities 
(e.g. ambulation, transportation, phone usage, daily activities, exercise, military);
the selection of different sensors to use (e.g. inertial, light, temperature or audio sensors)
\cite{Lara2013} and different bodily locations for sensor placement 
(e.g. chest, wrist, lower back, hip, thigh, foot) \cite{Cleland2013}
results in a large number of options for configurations \cite{Mannini2013}.

According to Bulling \emph{et al.}  \cite{bulling2014} the common challenges in HAR
using body-worn sensors are: 
(i) \textit{intraclass variability} which occurs when an activity is performed differently 
either by a single person or several people. For example, gait patterns may be more
dynamic in the morning after sleep than in the evening after a day full of activities; 
(ii) \textit{interclass similarity} occurs when the sensor data is very similar. For example,
in recognising dietary activity, drinking water or coffee entails the same arm movements 
\cite{amft2008phd};
and (iii) \textit{the NULL class problem} occurs 
when ambiguous activities are irrelevant for the recognition methods
which leads to wrong classification of the activities \cite{amft2011}.

For this PhD, the variability of dance activitities is a very rich case of study 
to explore and investigate the human movement variability.
Variability is presented in either dance features 
(e.g. fluency of motion, coordination, steadiness of the rhythm, 
adding erratic or additional movements \cite{Grammer2011, Aristidou2014}) 
or some biological and demographic features of dancers 
(e.g. gender, age, home country \cite{Grammer2011, Iwai2011}). 

Hammerla \emph{et al.} \cite{hammerla2011} have examined the effects of
variability of motor performance using artificial signals so as to create 
motion structures (strategy of the motion activity) and motion noise 
(the precision of the motion) of human activities. 
To quantify the variability of motion activities, Hammerla \emph{et al.} 
\cite{hammerla2011} proposed the use of PCA to compute the area underneath 
the cumulative energy curve which is used as a metric for motor skill assessment.
The variability in human activities has therefore a relation with quantitative assessment
of motion structures and motion noise of human activities.
Velloso \emph{et al.} \cite{Velloso2013a}, for example, assessed automatically the quality 
of weight-lifting activity.
Similarly, Velloso \emph{et al.} \cite{Velloso2013b} quantify how \textit{good} the repetition of 
weight-lifting activity is in terms of angles of each bone in relation to reference planes. 


Similarly, concepts from non-linear analysis such as fractal dimensionality, 
the Lyapunov exponent or time-delay embedding have been applied to better 
understand the variability of human activities.
For instance, Yamamoto \emph{et al.} \cite{Suzuki2013, Yamamoto2000} used the fractal dimensionality 
of the attractors to model repeated forehand and backhand tennis strokes.
Gouwanda \emph{et al.} \cite{Gouwanda2012} showed that the variability in walking speed 
has a linear relationship with the Lyapunov exponent. 
This exponent is therefore suitable 
for analysing the temporal variation in gait stability.
Time-delay embedding has been used as a feature 
for general gait recognition \cite{Sama2013} as well as for recognition 
of cycling, running, walking up stairs and downstairs activities \cite{Frank2010}.
Recently,  Caballero \emph{et al.} \cite{Caballero2014} reviewed 
further non-linear analysis tools
(e.g. local dynamic stability, recurrent quantification analysis, entropy measurements,
detrended fluctuation analysis) to measure human movement variability. 
However, the questions to ask, as pointed out by Caballero \emph{et al.} \cite{Caballero2014}, 
are: ``...do these tools actually measure variability?'' and ``what kind of variability?''.
It should be noted that non-linear analysis offers a range of techniques for the study of 
human activity (see \cite{Guastello2011} for an overview of alternative techniques). 

Given the case of the variability in dance activities, it is hypothesised that 
there are three possible reasons for variation:
(i) inherent noise in body-worn sensors, 
(ii) inherent properties of the activity itself and
(iii) discrepancies of biological features of people.

For this PhD, the following research questions will be addressed:
\begin{enumerate}
 \item How can the time-delay embedding and PCA methods quantify the possible reasons of 
 the variability of dance activities ?
 \item In the light of limitations of time-delay embedding and PCA,
 which other non-linear analysis tools would be suitable to explore 
 the variability in dancing activities and use them as a features for machine learning algorithms?
\end{enumerate}







\section{Recognising Dexterity In Dance }
As Miura et al. \cite{Miura2015} point out ''$\ldots$ how the human motor system produces 
dance movements is still poorly understood.`` A key issue concerns the manner in which experienced 
dancers solve the 'degrees of freedom' problem in face of changing contextual demands.  
Miura et al. \cite{Miura2013} measured muscle activation using electromyographic (EMG) data collected 
from muscles in the lower limb, for a task requiring participants to 
bounce up and down in time to a metronome beat.  
They demonstrated that experienced dancers show much better precision in synchronizing 
movements to beat than non-dancers, i.e., dancers maintained much lower standard 
deviation in temporal deviation against the beat than non-dancers. 
This result is consistent with work which shows that, compared with inexperienced- or 
non-dancers, trained ballet dancers exhibit superior postural 
stability \cite{Crotts1996}, and show superior ability in position matching 
of upper limbs \cite{Ramsay2001}.

Capturing dance activity through sensors has tended to rely on motion capture 
\cite{Alexiadis2014} or sensors mounted on the person \cite{Lynch2005} 
or in their shoes \cite{Paradiso1997} or data recorded from their smartphones \cite{Wei2014}.
Much of this work has been concerned with using the dancers' motion to work with 
multimedia presentations that augment and complement the dance \cite{Griffith1998, Park2006}
or as interfacing with a game \cite{Chu2012} or commercial games, such as Dance Dance Revolution.  
While the range of sensing technology used in these papers is diverse and the results 
of the activity recognition are varied, it is fair to say that few of the papers have considered 
variability or dexterity in how a dance is performed. 
In their work, Aristidou et al. \cite{Aristidou2014}
have considered the manner in which dance steps conform to a set of defined 
templates that describe steps in terms of a three-dimensional rotation (described using quaternions).  
The implication is that a goodness-of-fit can be ascertained to determine how good a dancer 
performs a step, and how any deviation from ‘good’ can be modified and improved through practice. 

For this report, we are interested in the question of how time-delay embedding techniques 
can provide insight into the variability and dexterity of dancers. 
To this end, we consider the performance of a set of steps from Salsa dance 
as well as other dance styles in future and 
compare untrained, inexperienced or non-dancers in one cohort with experienced dancers in another.

Before explaining how the activity recognition chain, the next section outlines the approach to time-series 
time-delay embedding and the resulting phase space representation used in this report. 



\section{Time-delay Embedding}
The aim of time-delay embedding, also known as Takens's Theorem, is to reconstruct 
a $k-$dimensional manifold $M$ of an unknown dynamical system $s(t)$ 
from a time series $x(t)$ with discrete observations at given timepoints $t$. 
Time-delay embedding assumes that the time series 
is a sequence $x(t)=h[s(t)]$,  where  $h: M \rightarrow \mathbb{R}$ 
is a measurement function in the unknown dynamical system, being $x(t)$ measurable.

Thus, the time delay reconstruction is defined as:
$\overline{x}(t) = (x(t), x(t-\tau),...,x(t-(m-1)\tau))$ 
where $m$ is the embedding dimension and $\tau$ is the embedding time-delay.
$\overline{x}(t)$ defines a map $\varPhi: M \rightarrow \mathbb{R}^m$ such that 
$\overline{x}(t) = \varPhi(s)$.
Similarly, $y(t)= \varPsi [\overline{x}(t)]$ is a $n$-dimensional vector 
where $\varPsi: \mathbb{R}^m \rightarrow \mathbb{R}^n$ is a further transformation 
(e.g., PCA \cite{Shlens2014}, Nonlinear PCA \cite{Kruger2007}, 
Locally Linear Embedding \cite{Roweis2000}). Figure \ref{fig:takens_theorem} 
illustrate the time delay reconstruction process.
For details, see the work of Uzal \emph{et al.} \cite{Uzal2011}.

\begin{figure}[!htb]
\centering    
 \includegraphics[width=0.45\textwidth]{takens_theorem_v6}
\caption[PA]{The reconstruction problem. The figure is based on the work of Uzal 
\emph{et al.} \cite{Uzal2011}.}
\label{fig:takens_theorem}
\end{figure}

\subsection{Embedding Parameters $m$ and $\tau$}
Given any time series $x(t)$, the time delay reconstruction system, $\overline{x}(t)$,
is easy to implement. For this work, Cao's method \cite{Cao1997}, a modification of the 
False Nearest Neighbours (FNN) algorithm, and mutual information algorithm by 
Fraser \emph{et al.} \cite{Fraser1986} have been used to calculate minimum embedding 
parameters ($m_{min}$, $\tau_{min}$).

\subsubsection{Minimum Embedding Dimension $m_{min}$}
Cao's method \cite{Cao1997} for computing the minimal embedding dimension is based on 
the mean values $E1(d)$ and $E2(d)$ in which $d$ is a given embedding dimension value.

$E1(d)$ is used to obtain the minimal dimension $m_{min}$ and stops changing 
when the time series comes from an attractor (Figure~\ref{fig:e1e2} B).
We computed $E1(d)$ values for $1 \leq \tau \leq 10$ to exemplify 
the minor dependency of $\tau$ given periodic, chaotic and random time series
(Figures~\ref{fig:e1e2} (A,B,C)).

The second of these values, $E2(d)$, is used to distinguish 
deterministic signals from random signals in which case the $E2(d)$ values will be approximately 
equal to 1 for any $d$ (Figure~\ref{fig:e1e2} F).
Similarly, we computed $E2(d)$ values for periodic, chaotic and random time series,
to exemplify the no significative dependency on $\tau$, where $1 \leq \tau \leq 10$ 
(Figures~\ref{fig:e1e2} (D,E,F)).

Cao's method is a modified version of the FNN method, and $E1(d)$ and 
$E2(d)$ values are only dependant on $m$ and $\tau$ \cite{Cao1997}.

% illustrate our application of this approach,
% which is described in more detail in the Estimation of the Minimal Embedded Parameters section.

\begin{figure}[!htb]
\centering    
 \includegraphics[width=0.45\textwidth]{e1e2_v00}
\caption[PA]{The values of $E1(d)$ and $E2(d)$ with different time delay embedding parameters
from periodic (A,D), chaotic (B,E) and random (C,F) time series.}
\label{fig:e1e2}
\end{figure}


\subsubsection{Minimum Time-delay Embedding  $\tau_{min}$}
The method of choosing the minimum Time-delay embedding, $\tau_{min}$, was proposed 
by Fraser \emph{et al.} \cite{Fraser1986} in which the first minimum of the mutual 
information graph is chosen to estimate the minimal time-delay embedding parameter.
For instance, Figure~\ref{fig:mi}  illustrates 
the mutual information from periodic, chaotic and random time series.
The local minimum for the Chaotic series in Figure~\ref{fig:mi} is $\tau_{min} = 18$.
On the other hand, for the periodic and random time series 
the mutual information plots have no local minimum 
and values are monotonically decreasing which means that $\tau_{min} = 1$ for both
(Figure~\ref{fig:mi}) \cite{Fraser1986}.

\begin{figure}[!htb]
\centering    
 \includegraphics[width=0.45\textwidth]{mi_values_v00}
\caption[PA]{Mutual information plots from periodic, chaotic and random time series.}
\label{fig:mi}
\end{figure}


\subsubsection{Embedding Parameters Setbacks}
Although the time-delay embedding method using inertial sensors has been used extensively 
in gait recognition \cite{Sama2013}, gait stability \cite{Gouwanda2012} and walking, 
running and cycling activities \cite{Frank2010},
some problems with the minimal embedding parameter estimation  
($m_{min}$ and $\tau_{min}$) still remain to be solved.


Sama \emph{et al.} \cite{Sama2013} and Gouwanda \emph{et al.} \cite{Gouwanda2012}
estimated the minimal embedded dimension ($m_{min}$) 
with the False Nearest Neighbours (FNN) method. However, Cao \cite{Cao1997} pointed out that 
the FNN algorithm introduces new parameters ($R_{tol}$ and $A_{tol}$) that lead 
to different results and  cannot differentiate random series from deterministic series. 
Frank \emph{et al.} \cite{Frank2010} proposed a grid search method to find the minimal 
embedded parameters, but there are no details about their approach.

In the case of the minimal time delay embedding value, $\tau_{min}$,
Fojt \emph{et al.} \cite{Fojt1998} mentioned a method in which the chosen $\tau$ 
is made in function of filling optimally the reconstructed state space;
however, Fojt \emph{et al.} \cite{Fojt1998} mentioned
that ``it is a rough estimation based on a graphical procedure.''
Although, Sama \emph{et al.} \cite{Sama2013} computed $\tau_{min}$ using 
the method proposed by Fraser \emph{et al.} \cite{Fraser1986}, 
they pointed that the chosen $\tau_{min}$ largely depend on the application.


% \section{Human Activity Recognition}
% 
% .............
% .............
% .............
% add intro
% .............
% .............
% .............
% \subsection{The Activity Recognition Chain}

\section{The Activity Recognition Chain}

\begin{figure*}
\centering    
 \includegraphics[width=\textwidth]{ARC02}
\caption[PA]{Typical activity recognition chain (ARC) to identify activities or gestures
from body-worn inertial sensors. 
Diagram is replicated from the work of Bulling \emph{et al.} \cite{bulling2014}.}
\label{fig:arc}
\end{figure*}

Bulling \emph{et al.} \cite{bulling2014} reviewed the state of the art of
HAR using body-worn inertial sensors.
Figure~\ref{fig:arc} illustrates the typical activity recognition chain (ARC) to identify
activities with body-worn sensors. 

The first stage of the ARC is the raw data collection from several sensors attached to 
different parts of the body. Sensors data over a given time, $s_i$, provide multiple values  $d^i$, 
(e.g. $d^1, d^2, d^3$ for 3-D acceleration referred to as x, y and z direction)
\begin{equation}
s_i = (\textbf{d}^1, \textbf{d}^2,\dots \textbf{d}^t) \mbox{,for } i=1, \dots,k
\end{equation} 
where $k$ denotes the number of sensors. 

In the preprocessing stage of the ARC, raw multivariable time series are transformed into a 
pre-processed time series $D'= (d'_1, \dots, d'_n )^T$, where $d'_i$ is one dimension 
of the data for the preprocessed time series and $n$ is the number of total data dimensions.
Different methods for the preprocessing tasks may be applied to the raw data
(e.g. synchronisation, calibration, unit conversion, normalisation, resampling, denoising 
or baseline drift removal \cite{bulling2014}).

The stage of data segmentation identifies segments within the continuous data stream
that are likely to have information about activities. The segmentation stage creates 
a set of segments $w_m$ % containing a possible activity $y$
such that 
\begin{equation}
W = \{   w_1, \dots, w_m  \},
\end{equation} 
where $m$ correspond to the number of segments.
Since the segmentation of the data is a difficult problem, there are various methods 
in the literature to tackle this problem: sliding window, energy-based segmentation, 
rest-position segmentation, additional sensors and external context sources.

In the feature extraction stage, a feature extraction function $F$ reduces
the signals $D'$  into segmented signals $W$.
The total number of features $X_i$ is the feature space.
\begin{equation}
X_i = F ( D', w_i)
\end{equation} 
In the literature on activity recognition, different  methods for feature extraction 
can be found including signal-based features, body model features, event-based features, 
multilevel features or automatic feature ranking and selection.


Machine learning tools have been used in HAR over the last 15 years
so as to describe, analyse and predict human activities \cite{bulling2014}.
However, the chosen approach is subject to computational complexity,
recognition performance or latency.
Generally for the learning stage, a training data set $T = \{ X_i, y_i \}  ^N _ {i=1}$ 
is computed prior to the classification with $N$ pairs of feature vectors $X_i$ and ground 
truth labels $y^i$ (possible activities to recognise). For this stage, model parameters $\theta$ can be learned 
to decrease the classification error on $T$. 
Then, with the trained model $T$,
each feature vector $X_i$  is mapped to a set of class labels 
$Y= \{ y^1, \dots , y^c \}$ with scores $P_i = \{ p^1_i, \dots, p^c_i \}$:
\begin{equation}
p_i ( y \mid X_i, \theta) = I (X_i, \theta) \mbox{,for } y \in Y
\end{equation} 
and inference method $I$.
Finally, the classification output $y_i$ is computed with the maximum score $P_i$
\begin{equation}
y_i  = 
\underset{ y \in Y, p \in P_i }{\text{argmax}}   p(y | X_i, \theta)
\end{equation} 
The most common classification algorithms are: decision trees, Bayesian models, %instance base,
domain transform, fuzzy logic, Markov models, support vector machines (SVM), 
artificial neural networks (ANN) and ensembles of classifiers \cite{Lara2013}.




% 
% Supervised learning also knows as classification.
% 
% 
% 
% Semi-supervised learning methods are useful when large amounts
% of activities change very often or some of the labels are missing.
% To this end, four methods have been proposed: multi-graphs,
% en-co-training, ali and huynh. However,
% this methods are still far from real HAR
% due to the nonstandard algorithms and limitaions such as computational
% complexity, no significative improvement in classification accuracy
% or very specific applications.
% LARA
% 

Similarly, when the recognition of activities can miss, confuse or falsely
recognise activities that did occur, several metrics
can be used to optimise the classification. Some of the metrics
are confusion matrices, accuracy, precision, recall, and F-scores,
decision-independent Precision-Recall or receiver operating characteristic
curves (ROC curves) \cite{bulling2014}.



% 
% 
% 
% Lara \emph{et al.}  \cite{Lara2013} proposed a taxonomy for HAR systems (Figure~\ref{fig:tHAR}),
% in which the learning approach of the wearable sensing can be either semi-supervised or supervised.
% Being supervised learning, based on the reposense time, either online or offline.
% 
% \begin{figure}[!htb]
% \centering    
%  \includegraphics[width=0.45\textwidth]{tHAR00}
% \caption[PA]{Taxonomy of Human Activity Recognition Systems \cite{Lara2013}.}
% \label{fig:tHAR}
% \end{figure}
% 


% \section{Methods}
\section{Artificial Signals}

Following the proposal of Hammerla \emph{et al.} \cite{hammerla2011},
artificial signals are created to examine the effects of
variability in the precision of motion  (additive noise) 
and in the strategy of motion  (structural noise) of 
activities.

Additive noise is normalised noise with $\sigma_a ^2$ added to the
sinusoid signal $S$: 
\begin{equation}
 S^a = S + \textbf{N}(0, \sigma_a ^2)
\end{equation} 

Structural noise is a sinusoid signal distorted with different variance in frequency and amplitude
$\sigma_s ^2$ and window length $w_s$. Algorithm 1 describes the creation of structural noise.
To make the data less redundant for possible variations of environmental 
conditions or body-worn sensor mobility in users, the data is whitened 
(i.e. data is normalised to have zero mean and unit variance) 


% http://tex.stackexchange.com/questions/219816/algorithm-in-ieee-format
\begin{algorithm}[H]
\caption{Structural Noise}
\begin{algorithmic}[1]
 \renewcommand{\algorithmicrequire}{\textbf{Input:}}
 \renewcommand{\algorithmicensure}{\textbf{Output:}}
 \REQUIRE time-series $S^a$, variance $\sigma_s ^2$, window length $w_s$
 \ENSURE  Structurally distorted signal $S^s$
  \FOR {$j = 1$ to $L$, $j=j+w_s$}
  \STATE $\textbf{u'} \leftarrow \textbf{N}(0, \sigma_s ^2)$
  \STATE $S^{a} =$ sinusoid with frequency $| \textbf{u'} |$ and variance $\sigma_a ^2$ of length $w_s$
  \STATE $S^s_{j \rightarrow j+w_s}  = S^s_{j \rightarrow j+w_s}  + S^{a} \times   \sigma_s ^2 $
  \ENDFOR
  \\ $S^s= whiten (S^s)$
 \RETURN $S^s$
\end{algorithmic}
\end{algorithm}

By varying both $\sigma_a ^2$ and $\sigma_s ^2$ 
is possible to simulate and control the additive noise and the structural noise 
in the structure of the human activity. For example, low values of $\sigma_a ^2$ 
are associated with precise movements while low values of $\sigma_s ^2$ correspond 
to a well chosen strategy for a motion.


\begin{figure}[!htb]
\centering    
 \includegraphics[width=0.45\textwidth]{impactofnoise01}
\caption[PA]{Graphs (A, B, C and D) present the variability of additive noise 
($\sigma_a ^2$) and structural noise ($\sigma_s ^2$ ) on the sinusoid signals with $w_s=500$ 
according to the parameters indicated on the left plot.}
\label{fig:sn}
\end{figure}

For graphs in Figure~\ref{fig:sn}, the base frequency of the sinusoid is 1 Hz sampled at 50Hz with an 
amplitude 1 and window length of 250.





\section{WORKING ON hammerla method to takenspca}
\section{WORKING ON results with nondancers data}



\section{Publication Plan}

Publish my research's results in Measuring Behaviour 2016
and Augmented Human 2016 conferences so as to push myself 
to submit a journal in the 15th Month of the PhD.


\section{Conclusion and Future Work} 
Preliminary results were submitted to ISWC 2016; however, the 
submission was rejected because the reviewers argued that 
the method of using the reconstructed state space 
to quantify dexterity for salsa dancers is too specific and
it is not transferible to other applications. Additionally, 
data analyis were collected from 11 novices, 1 intermediate and 1 expert dancers,
and no classification framework was performed. 

% For future experiments, the experiment' conditions should be highly customisable and repeatable. 

To raise the bar in the field of human activity recognition,
we are planning to collect more data from experts dancers so as to 
capture as much of the variability as possible in different dancing tasks.
Classify our data according to the better feature representation as well as 
the evaluation of the approach for different variation of activities such as 
tool usage, sport skills to mention but a few.





% This report is organised as follow.

% \hfill mds
% \hfill September 17, 2014

%%%%%%%%%%%%%%%%%%%%%%%%%%%%%%%%%%%%%%%%%%
% WHY IS THE IMPORTANCE OF USING TDE?
% ADD ANTOHER REFEREN CES OF THE TIME DELAY EMBEDDING 
% Human body analysis using the time-delay embedding 
% \cite{Fojt1998}

% \subsection{Subsection Heading Here}
% Subsection text here.
% 
% % needed in second column of first page if using \IEEEpubid
% %\IEEEpubidadjcol
% 


% Note that IEEE typically puts floats only at the top, even when this
% results in a large percentage of a column being occupied by floats.
% However, the Computer Society has been known to put floats at the bottom.


% An example of a double column floating figure using two subfigures.
% (The subfig.sty package must be loaded for this to work.)
% The subfigure \label commands are set within each subfloat command,
% and the \label for the overall figure must come after \caption.
% \hfil is used as a separator to get equal spacing.
% Watch out that the combined width of all the subfigures on a 
% line do not exceed the text width or a line break will occur.
%
%\begin{figure*}[!t]
%\centering
%\subfloat[Case I]{\includegraphics[width=2.5in]{box}%
%\label{fig_first_case}}
%\hfil
%\subfloat[Case II]{\includegraphics[width=2.5in]{box}%
%\label{fig_second_case}}
%\caption{Simulation results for the network.}
%\label{fig_sim}
%\end{figure*}
%
% Note that often IEEE papers with subfigures do not employ subfigure
% captions (using the optional argument to \subfloat[]), but instead will
% reference/describe all of them (a), (b), etc., within the main caption.
% Be aware that for subfig.sty to generate the (a), (b), etc., subfigure
% labels, the optional argument to \subfloat must be present. If a
% subcaption is not desired, just leave its contents blank,
% e.g., \subfloat[].


% An example of a floating table. Note that, for IEEE style tables, the
% \caption command should come BEFORE the table and, given that table
% captions serve much like titles, are usually capitalized except for words
% such as a, an, and, as, at, but, by, for, in, nor, of, on, or, the, to
% and up, which are usually not capitalized unless they are the first or
% last word of the caption. Table text will default to \footnotesize as
% IEEE normally uses this smaller font for tables.
% The \label must come after \caption as always.
%
%\begin{table}[!t]
%% increase table row spacing, adjust to taste
%\renewcommand{\arraystretch}{1.3}
% if using array.sty, it might be a good idea to tweak the value of
% \extrarowheight as needed to properly center the text within the cells
%\caption{An Example of a Table}
%\label{table_example}
%\centering
%% Some packages, such as MDW tools, offer better commands for making tables
%% than the plain LaTeX2e tabular which is used here.
%\begin{tabular}{|c||c|}
%\hline
%One & Two\\
%\hline
%Three & Four\\
%\hline
%\end{tabular}
%\end{table}


% Note that the IEEE does not put floats in the very first column
% - or typically anywhere on the first page for that matter. Also,
% in-text middle ("here") positioning is typically not used, but it
% is allowed and encouraged for Computer Society conferences (but
% not Computer Society journals). Most IEEE journals/conferences use
% top floats exclusively. 
% Note that, LaTeX2e, unlike IEEE journals/conferences, places
% footnotes above bottom floats. This can be corrected via the
% \fnbelowfloat command of the stfloats package.



% \section{Conclusion}
% The conclusion goes here.
% 
% % if have a single appendix:
% %\appendix[Proof of the Zonklar Equations]
% % or
% %\appendix  % for no appendix heading
% % do not use \section anymore after \appendix, only \section*
% % is possibly needed
% 
% % use appendices with more than one appendix
% % then use \section to start each appendix
% % you must declare a \section before using any
% % \subsection or using \label (\appendices by itself
% % starts a section numbered zero.)
% %
% 
% \appendices
% \section{Proof of the First Zonklar Equation}
% Appendix one text goes here.
% 
% % you can choose not to have a title for an appendix
% % if you want by leaving the argument blank
% \section{}
% Appendix two text goes here.





% use section* for acknowledgment
\ifCLASSOPTIONcompsoc
  % The Computer Society usually uses the plural form
  \section*{Acknowledgments}
\else
  % regular IEEE prefers the singular form
  \section*{Acknowledgment}
\fi

Miguel P\'erez-Xochicale gratefully acknowledges the studentship from 
the National Council for Science and Technology (CONACyT) Mexico
to pursue his postgraduate studies at University of Birmingham.

% Can use something like this to put references on a page
% by themselves when using endfloat and the captionsoff option.
\ifCLASSOPTIONcaptionsoff
  \newpage
\fi



% trigger a \newpage just before the given reference
% number - used to balance the columns on the last page
% adjust value as needed - may need to be readjusted if
% the document is modified later
%\IEEEtriggeratref{8}
% The "triggered" command can be changed if desired:
%\IEEEtriggercmd{\enlargethispage{-5in}}

% references section

% can use a bibliography generated by BibTeX as a .bbl file
% BibTeX documentation can be easily obtained at:
% http://www.ctan.org/tex-archive/biblio/bibtex/contrib/doc/
% The IEEEtran BibTeX style support page is at:
% http://www.michaelshell.org/tex/ieeetran/bibtex/
%\bibliographystyle{IEEEtran}
% argument is your BibTeX string definitions and bibliography database(s)
%\bibliography{IEEEabrv,../bib/paper}
%
% <OR> manually copy in the resultant .bbl file
% set second argument of \begin to the number of references
% (used to reserve space for the reference number labels box)
% \begin{thebibliography}{1}
% 
% \bibitem{IEEEhowto:kopka}
% H.~Kopka and P.~W. Daly, \emph{A Guide to \LaTeX}, 3rd~ed.\hskip 1em plus
%   0.5em minus 0.4em\relax Harlow, England: Addison-Wesley, 1999.
% 
% \end{thebibliography}

% \nocite{*}
\bibliographystyle{IEEEtran}
\bibliography{references}


% biography section
% 
% If you have an EPS/PDF photo (graphicx package needed) extra braces are
% needed around the contents of the optional argument to biography to prevent
% the LaTeX parser from getting confused when it sees the complicated
% \includegraphics command within an optional argument. (You could create
% your own custom macro containing the \includegraphics command to make things
% simpler here.)
%\begin{IEEEbiography}[{\includegraphics[width=1in,height=1.25in,clip,keepaspectratio]{mshell}}]{Michael Shell}
% or if you just want to reserve a space for a photo:

% \begin{IEEEbiography}[{\includegraphics[width=1in,height=1.25in,clip,keepaspectratio]{mxochicale38x44.pdf}}]{name}

% \begin{IEEEbiography}{Miguel Perez-Xochicale}
% ........................
% \end{IEEEbiography}



% % if you will not have a photo at all:
% \begin{IEEEbiographynophoto}{John Doe}
% Biography text here.
% \end{IEEEbiographynophoto}
% 
% % insert where needed to balance the two columns on the last page with
% % biographies
% %\newpage
% 
% \begin{IEEEbiographynophoto}{Jane Doe}
% Biography text here.
% \end{IEEEbiographynophoto}

% You can push biographies down or up by placing
% a \vfill before or after them. The appropriate
% use of \vfill depends on what kind of text is
% on the last page and whether or not the columns
% are being equalized.

%\vfill

% Can be used to pull up biographies so that the bottom of the last one
% is flush with the other column.
%\enlargethispage{-5in}



% that's all folks
\end{document}
