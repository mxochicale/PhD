%% bare_jrnl_compsoc.tex
%% V1.4a
%% 2014/09/17
%% by Michael Shell
%% See:
%% http://www.michaelshell.org/
%% for current contact information.
%%
%% This is a skeleton file demonstrating the use of IEEEtran.cls
%% (requires IEEEtran.cls version 1.8a or later) with an IEEE
%% Computer Society journal paper.
%%
%% Support sites:
%% http://www.michaelshell.org/tex/ieeetran/
%% http://www.ctan.org/tex-archive/macros/latex/contrib/IEEEtran/
%% and
%% http://www.ieee.org/

%%*************************************************************************
%% Legal Notice:
%% This code is offered as-is without any warranty either expressed or
%% implied; without even the implied warranty of MERCHANTABILITY or
%% FITNESS FOR A PARTICULAR PURPOSE! 
%% User assumes all risk.
%% In no event shall IEEE or any contributor to this code be liable for
%% any damages or losses, including, but not limited to, incidental,
%% consequential, or any other damages, resulting from the use or misuse
%% of any information contained here.
%%
%% All comments are the opinions of their respective authors and are not
%% necessarily endorsed by the IEEE.
%%
%% This work is distributed under the LaTeX Project Public License (LPPL)
%% ( http://www.latex-project.org/ ) version 1.3, and may be freely used,
%% distributed and modified. A copy of the LPPL, version 1.3, is included
%% in the base LaTeX documentation of all distributions of LaTeX released
%% 2003/12/01 or later.
%% Retain all contribution notices and credits.
%% ** Modified files should be clearly indicated as such, including  **
%% ** renaming them and changing author support contact information. **
%%
%% File list of work: IEEEtran.cls, IEEEtran_HOWTO.pdf, bare_adv.tex,
%%                    bare_conf.tex, bare_jrnl.tex, bare_conf_compsoc.tex,
%%                    bare_jrnl_compsoc.tex, bare_jrnl_transmag.tex
%%*************************************************************************


% *** Authors should verify (and, if needed, correct) their LaTeX system  ***
% *** with the testflow diagnostic prior to trusting their LaTeX platform ***
% *** with production work. IEEE's font choices and paper sizes can       ***
% *** trigger bugs that do not appear when using other class files.       ***
% The testflow support page is at:
% http://www.michaelshell.org/tex/testflow/


% \documentclass[10pt,journal,compsoc]{IEEEtran}
\documentclass[9pt,journal,onecolumn,compsoc]{IEEEtran}


 
 
%
% If IEEEtran.cls has not been installed into the LaTeX system files,
% manually specify the path to it like:
% \documentclass[10pt,journal,compsoc]{../sty/IEEEtran}



\usepackage[switch]{lineno}
\linenumbers
\leftlinenumbers
% http://tex.stackexchange.com/questions/20368/lineno-for-2-column


\usepackage{eurosym}

% Some very useful LaTeX packages include:
% (uncomment the ones you want to load)


% *** MISC UTILITY PACKAGES ***
%
%\usepackage{ifpdf}
% Heiko Oberdiek's ifpdf.sty is very useful if you need conditional
% compilation based on whether the output is pdf or dvi.
% usage:
% \ifpdf
%   % pdf code
% \else
%   % dvi code
% \fi
% The latest version of ifpdf.sty can be obtained from:
% http://www.ctan.org/tex-archive/macros/latex/contrib/oberdiek/
% Also, note that IEEEtran.cls V1.7 and later provides a builtin
% \ifCLASSINFOpdf conditional that works the same way.
% When switching from latex to pdflatex and vice-versa, the compiler may
% have to be run twice to clear warning/error messages.






% *** CITATION PACKAGES ***
%
\ifCLASSOPTIONcompsoc
  % IEEE Computer Society needs nocompress option
  % requires cite.sty v4.0 or later (November 2003)
  \usepackage[nocompress]{cite}
\else
  % normal IEEE
  \usepackage{cite}
\fi
% cite.sty was written by Donald Arseneau
% V1.6 and later of IEEEtran pre-defines the format of the cite.sty package
% \cite{} output to follow that of IEEE. Loading the cite package will
% result in citation numbers being automatically sorted and properly
% "compressed/ranged". e.g., [1], [9], [2], [7], [5], [6] without using
% cite.sty will become [1], [2], [5]--[7], [9] using cite.sty. cite.sty's
% \cite will automatically add leading space, if needed. Use cite.sty's
% noadjust option (cite.sty V3.8 and later) if you want to turn this off
% such as if a citation ever needs to be enclosed in parenthesis.
% cite.sty is already installed on most LaTeX systems. Be sure and use
% version 5.0 (2009-03-20) and later if using hyperref.sty.
% The latest version can be obtained at:
% http://www.ctan.org/tex-archive/macros/latex/contrib/cite/
% The documentation is contained in the cite.sty file itself.
%
% Note that some packages require special options to format as the Computer
% Society requires. In particular, Computer Society  papers do not use
% compressed citation ranges as is done in typical IEEE papers
% (e.g., [1]-[4]). Instead, they list every citation separately in order
% (e.g., [1], [2], [3], [4]). To get the latter we need to load the cite
% package with the nocompress option which is supported by cite.sty v4.0
% and later. Note also the use of a CLASSOPTION conditional provided by
% IEEEtran.cls V1.7 and later.


  


% *** GRAPHICS RELATED PACKAGES ***
%
\ifCLASSINFOpdf
  \usepackage[pdftex]{graphicx}
   \graphicspath{{figures/}}
  % declare the path(s) where your graphic files are
  % \graphicspath{{../pdf/}{../jpeg/}}
  % and their extensions so you won't have to specify these with
  % every instance of \includegraphics
  % \DeclareGraphicsExtensions{.pdf,.jpeg,.png}
\else
  % or other class option (dvipsone, dvipdf, if not using dvips). graphicx
  % will default to the driver specified in the system graphics.cfg if no
  % driver is specified.
  % \usepackage[dvips]{graphicx}
  % declare the path(s) where your graphic files are
  % \graphicspath{{../eps/}}
  % and their extensions so you won't have to specify these with
  % every instance of \includegraphics
  % \DeclareGraphicsExtensions{.eps}
\fi
% graphicx was written by David Carlisle and Sebastian Rahtz. It is
% required if you want graphics, photos, etc. graphicx.sty is already
% installed on most LaTeX systems. The latest version and documentation
% can be obtained at: 
% http://www.ctan.org/tex-archive/macros/latex/required/graphics/
% Another good source of documentation is "Using Imported Graphics in
% LaTeX2e" by Keith Reckdahl which can be found at:
% http://www.ctan.org/tex-archive/info/epslatex/
%
% latex, and pdflatex in dvi mode, support graphics in encapsulated
% postscript (.eps) format. pdflatex in pdf mode supports graphics
% in .pdf, .jpeg, .png and .mps (metapost) formats. Users should ensure
% that all non-photo figures use a vector format (.eps, .pdf, .mps) and
% not a bitmapped formats (.jpeg, .png). IEEE frowns on bitmapped formats
% which can result in "jaggedy"/blurry rendering of lines and letters as
% well as large increases in file sizes.
%
% You can find documentation about the pdfTeX application at:
% http://www.tug.org/applications/pdftex






% *** MATH PACKAGES ***
%
\usepackage[cmex10]{amsmath}
% A popular package from the American Mathematical Society that provides
% many useful and powerful commands for dealing with mathematics. If using
% it, be sure to load this package with the cmex10 option to ensure that
% only type 1 fonts will utilized at all point sizes. Without this option,
% it is possible that some math symbols, particularly those within
% footnotes, will be rendered in bitmap form which will result in a
% document that can not be IEEE Xplore compliant!
%
% Also, note that the amsmath package sets \interdisplaylinepenalty to 10000
% thus preventing page breaks from occurring within multiline equations. Use:
%\interdisplaylinepenalty=2500
% after loading amsmath to restore such page breaks as IEEEtran.cls normally
% does. amsmath.sty is already installed on most LaTeX systems. The latest
% version and documentation can be obtained at:
% http://www.ctan.org/tex-archive/macros/latex/required/amslatex/math/


\usepackage{amsfonts} % to use $\mathbb{Z}$


% *** SPECIALIZED LIST PACKAGES ***

\usepackage{algorithm}
\usepackage{algorithmic}
% algorithmic.sty was written by Peter Williams and Rogerio Brito.
% This package provides an algorithmic environment fo describing algorithms.
% You can use the algorithmic environment in-text or within a figure
% environment to provide for a floating algorithm. Do NOT use the algorithm
% floating environment provided by algorithm.sty (by the same authors) or
% algorithm2e.sty (by Christophe Fiorio) as IEEE does not use dedicated
% algorithm float types and packages that provide these will not provide
% correct IEEE style captions. The latest version and documentation of
% algorithmic.sty can be obtained at:
% http://www.ctan.org/tex-archive/macros/latex/contrib/algorithms/
% There is also a support site at:
% http://algorithms.berlios.de/index.html
% Also of interest may be the (relatively newer and more customizable)
% algorithmicx.sty package by Szasz Janos:
% http://www.ctan.org/tex-archive/macros/latex/contrib/algorithmicx/




% *** ALIGNMENT PACKAGES ***
%
%\usepackage{array}
% Frank Mittelbach's and David Carlisle's array.sty patches and improves
% the standard LaTeX2e array and tabular environments to provide better
% appearance and additional user controls. As the default LaTeX2e table
% generation code is lacking to the point of almost being broken with
% respect to the quality of the end results, all users are strongly
% advised to use an enhanced (at the very least that provided by array.sty)
% set of table tools. array.sty is already installed on most systems. The
% latest version and documentation can be obtained at:
% http://www.ctan.org/tex-archive/macros/latex/required/tools/


% IEEEtran contains the IEEEeqnarray family of commands that can be used to
% generate multiline equations as well as matrices, tables, etc., of high
% quality.




% *** SUBFIGURE PACKAGES ***
%\ifCLASSOPTIONcompsoc
%  \usepackage[caption=false,font=footnotesize,labelfont=sf,textfont=sf]{subfig}
%\else
%  \usepackage[caption=false,font=footnotesize]{subfig}
%\fi
% subfig.sty, written by Steven Douglas Cochran, is the modern replacement
% for subfigure.sty, the latter of which is no longer maintained and is
% incompatible with some LaTeX packages including fixltx2e. However,
% subfig.sty requires and automatically loads Axel Sommerfeldt's caption.sty
% which will override IEEEtran.cls' handling of captions and this will result
% in non-IEEE style figure/table captions. To prevent this problem, be sure
% and invoke subfig.sty's "caption=false" package option (available since
% subfig.sty version 1.3, 2005/06/28) as this is will preserve IEEEtran.cls
% handling of captions.
% Note that the Computer Society format requires a sans serif font rather
% than the serif font used in traditional IEEE formatting and thus the need
% to invoke different subfig.sty package options depending on whether
% compsoc mode has been enabled.
%
% The latest version and documentation of subfig.sty can be obtained at:
% http://www.ctan.org/tex-archive/macros/latex/contrib/subfig/




% *** FLOAT PACKAGES ***
%
%\usepackage{fixltx2e}
% fixltx2e, the successor to the earlier fix2col.sty, was written by
% Frank Mittelbach and David Carlisle. This package corrects a few problems
% in the LaTeX2e kernel, the most notable of which is that in current
% LaTeX2e releases, the ordering of single and double column floats is not
% guaranteed to be preserved. Thus, an unpatched LaTeX2e can allow a
% single column figure to be placed prior to an earlier double column
% figure. The latest version and documentation can be found at:
% http://www.ctan.org/tex-archive/macros/latex/base/


%\usepackage{stfloats}
% stfloats.sty was written by Sigitas Tolusis. This package gives LaTeX2e
% the ability to do double column floats at the bottom of the page as well
% as the top. (e.g., "\begin{figure*}[!b]" is not normally possible in
% LaTeX2e). It also provides a command:
%\fnbelowfloat
% to enable the placement of footnotes below bottom floats (the standard
% LaTeX2e kernel puts them above bottom floats). This is an invasive package
% which rewrites many portions of the LaTeX2e float routines. It may not work
% with other packages that modify the LaTeX2e float routines. The latest
% version and documentation can be obtained at:
% http://www.ctan.org/tex-archive/macros/latex/contrib/sttools/
% Do not use the stfloats baselinefloat ability as IEEE does not allow
% \baselineskip to stretch. Authors submitting work to the IEEE should note
% that IEEE rarely uses double column equations and that authors should try
% to avoid such use. Do not be tempted to use the cuted.sty or midfloat.sty
% packages (also by Sigitas Tolusis) as IEEE does not format its papers in
% such ways.
% Do not attempt to use stfloats with fixltx2e as they are incompatible.
% Instead, use Morten Hogholm'a dblfloatfix which combines the features
% of both fixltx2e and stfloats:
%
% \usepackage{dblfloatfix}
% The latest version can be found at:
% http://www.ctan.org/tex-archive/macros/latex/contrib/dblfloatfix/




%\ifCLASSOPTIONcaptionsoff
%  \usepackage[nomarkers]{endfloat}
% \let\MYoriglatexcaption\caption
% \renewcommand{\caption}[2][\relax]{\MYoriglatexcaption[#2]{#2}}
%\fi
% endfloat.sty was written by James Darrell McCauley, Jeff Goldberg and 
% Axel Sommerfeldt. This package may be useful when used in conjunction with 
% IEEEtran.cls'  captionsoff option. Some IEEE journals/societies require that
% submissions have lists of figures/tables at the end of the paper and that
% figures/tables without any captions are placed on a page by themselves at
% the end of the document. If needed, the draftcls IEEEtran class option or
% \CLASSINPUTbaselinestretch interface can be used to increase the line
% spacing as well. Be sure and use the nomarkers option of endfloat to
% prevent endfloat from "marking" where the figures would have been placed
% in the text. The two hack lines of code above are a slight modification of
% that suggested by in the endfloat docs (section 8.4.1) to ensure that
% the full captions always appear in the list of figures/tables - even if
% the user used the short optional argument of \caption[]{}.
% IEEE papers do not typically make use of \caption[]'s optional argument,
% so this should not be an issue. A similar trick can be used to disable
% captions of packages such as subfig.sty that lack options to turn off
% the subcaptions:
% For subfig.sty:
% \let\MYorigsubfloat\subfloat
% \renewcommand{\subfloat}[2][\relax]{\MYorigsubfloat[]{#2}}
% However, the above trick will not work if both optional arguments of
% the \subfloat command are used. Furthermore, there needs to be a
% description of each subfigure *somewhere* and endfloat does not add
% subfigure captions to its list of figures. Thus, the best approach is to
% avoid the use of subfigure captions (many IEEE journals avoid them anyway)
% and instead reference/explain all the subfigures within the main caption.
% The latest version of endfloat.sty and its documentation can obtained at:
% http://www.ctan.org/tex-archive/macros/latex/contrib/endfloat/
%
% The IEEEtran \ifCLASSOPTIONcaptionsoff conditional can also be used
% later in the document, say, to conditionally put the References on a 
% page by themselves.




% *** PDF, URL AND HYPERLINK PACKAGES ***
%
\usepackage{url}
% url.sty was written by Donald Arseneau. It provides better support for
% handling and breaking URLs. url.sty is already installed on most LaTeX
% systems. The latest version and documentation can be obtained at:
% http://www.ctan.org/tex-archive/macros/latex/contrib/url/
% Basically, \url{my_url_here}.





% *** Do not adjust lengths that control margins, column widths, etc. ***
% *** Do not use packages that alter fonts (such as pslatex).         ***
% There should be no need to do such things with IEEEtran.cls V1.6 and later.
% (Unless specifically asked to do so by the journal or conference you plan
% to submit to, of course. )


% correct bad hyphenation here
\hyphenation{op-tical net-works semi-conduc-tor  Birmingham}


\begin{document}
%
% paper title
% Titles are generally capitalized except for words such as a, an, and, as,
% at, but, by, for, in, nor, of, on, or, the, to and up, which are usually
% not capitalized unless they are the first or last word of the title.
% Linebreaks \\ can be used within to get better formatting as desired.
% Do not put math or special symbols in the title.
% \title{Bare Demo of IEEEtran.cls\\ for Computer Society Journals}
\title{
  Automatic Classification of  Human Movement Variability  in the context of  Human-Robot Interaction. (DG-B)
}



% author names and IEEE memberships
% note positions of commas and nonbreaking spaces ( ~ ) LaTeX will not break
% a structure at a ~ so this keeps an author's name from being broken across
% two lines.
% use \thanks{} to gain access to the first footnote area
% a separate \thanks must be used for each paragraph as LaTeX2e's \thanks
% was not built to handle multiple paragraphs
%
%
%\IEEEcompsocitemizethanks is a special \thanks that produces the bulleted
% lists the Computer Society journals use for "first footnote" author
% affiliations. Use \IEEEcompsocthanksitem which works much like \item
% for each affiliation group. When not in compsoc mode,
% \IEEEcompsocitemizethanks becomes like \thanks and
% \IEEEcompsocthanksitem becomes a line break with idention. This
% facilitates dual compilation, although admittedly the differences in the
% desired content of \author between the different types of papers makes a
% one-size-fits-all approach a daunting prospect. For instance, compsoc 
% journal papers have the author affiliations above the "Manuscript
% received ..."  text while in non-compsoc journals this is reversed. Sigh.

\author{
% Miguel Xochicale
 	Miguel~Xochicale,~\IEEEmembership{Doctoral~Researcher;}\\
         Chris~Baber,~\IEEEmembership{Lead~Supervisor;}
         ~Martin~Russell,~\IEEEmembership{Co-Supervisor;} \\
         and~?~?,~\IEEEmembership{Academic~Advisor.}
         % <-this % stops a space

 \IEEEcompsocitemizethanks{\IEEEcompsocthanksitem 
 M. Xochicale, C. Baber and M. Russell are with the School of Engineering, 
 The University of Birmingham, U.K. \protect\\
 % note need leading \protect in front of \\ to get a newline within \thanks as
 % \\ is fragile and will error, could use \hfil\break instead.
 E-mail: see http://mxochicale.github.io/
 }% <-this 
% stops an unwanted space
\thanks{Manuscript received September 2, 2016; revised Month Day, 2016.}
}

% note the % following the last \IEEEmembership and also \thanks - 
% these prevent an unwanted space from occurring between the last author name
% and the end of the author line. i.e., if you had this:
% 
% \author{....lastname \thanks{...} \thanks{...} }
%                     ^------------^------------^----Do not want these spaces!
%
% a space would be appended to the last name and could cause every name on that
% line to be shifted left slightly. This is one of those "LaTeX things". For
% instance, "\textbf{A} \textbf{B}" will typeset as "A B" not "AB". To get
% "AB" then you have to do: "\textbf{A}\textbf{B}"
% \thanks is no different in this regard, so shield the last } of each \thanks
% that ends a line with a % and do not let a space in before the next \thanks.
% Spaces after \IEEEmembership other than the last one are OK (and needed) as
% you are supposed to have spaces between the names. For what it is worth,
% this is a minor point as most people would not even notice if the said evil
% space somehow managed to creep in.

% The paper headers
\markboth{Twenty-First Month Report, August~2016}%
{Shell \MakeLowercase{\textit{et al.}}: Bare Demo of IEEEtran.cls for Computer 
Society Journals}
% The only time the second header will appear is for the odd numbered pages
% after the title page when using the twoside option.
% 
% *** Note that you probably will NOT want to include the author's ***
% *** name in the headers of peer review papers.                   ***
% You can use \ifCLASSOPTIONpeerreview for conditional compilation here if
% you desire.

% The publisher's ID mark at the bottom of the page is less important with
% Computer Society journal papers as those publications place the marks
% outside of the main text columns and, therefore, unlike regular IEEE
% journals, the available text space is not reduced by their presence.
% If you want to put a publisher's ID mark on the page you can do it like
% this:
%\IEEEpubid{0000--0000/00\$00.00~\copyright~2014 IEEE}
% or like this to get the Computer Society new two part style.
%\IEEEpubid{\makebox[\columnwidth]{\hfill 0000--0000/00/\$00.00~\copyright~2014 IEEE}%
%\hspace{\columnsep}\makebox[\columnwidth]{Published by the IEEE Computer Society\hfill}}
% Remember, if you use this you must call \IEEEpubidadjcol in the second
% column for its text to clear the IEEEpubid mark (Computer Society jorunal
% papers don't need this extra clearance.)

% use for special paper notices
%\IEEEspecialpapernotice{(Invited Paper)}

% for Computer Society papers, we must declare the abstract and index terms
% PRIOR to the title within the \IEEEtitleabstractindextext IEEEtran
% command as these need to go into the title area created by \maketitle.
% As a general rule, do not put math, special symbols or citations
% in the abstract or keywords.
\IEEEtitleabstractindextext{%


 
% \begin{abstract}
% ...
% \end{abstract}


% % Note that keywords are not normally used for peerreview papers.
% \begin{IEEEkeywords}
% X;X;X;
% \end{IEEEkeywords}



}



% make the title area
\maketitle


% To allow for easy dual compilation without having to reenter the
% abstract/keywords data, the \IEEEtitleabstractindextext text will
% not be used in maketitle, but will appear (i.e., to be "transported")
% here as \IEEEdisplaynontitleabstractindextext when the compsoc 
% or transmag modes are not selected <OR> if conference mode is selected 
% - because all conference papers position the abstract like regular
% papers do.
\IEEEdisplaynontitleabstractindextext
% \IEEEdisplaynontitleabstractindextext has no effect when using
% compsoc or transmag under a non-conference mode.



% For peer review papers, you can put extra information on the cover
% page as needed:
% \ifCLASSOPTIONpeerreview
% \begin{center} \bfseries EDICS Category: 3-BBND \end{center}
% \fi
%
% For peerreview papers, this IEEEtran command inserts a page break and
% creates the second title. It will be ignored for other modes.
\IEEEpeerreviewmaketitle





\IEEEraisesectionheading{\section{Introduction}\label{sec:introduction}}


This report presents the progress of the PhD research project titled
``Automatic Classification of Human Movement Variability in the context of Human-Robot Interaction'' 
for the period between September 2015 and August 2016.


\subsection{Research Questions}
As 
I have been exploring the area of Human Activity Recognition (HAR)
using on-body sensors to measure the variability of human movement 
and 
with the need to automatically assess simple and repetitive actions 
within and between subjects in different seasons.
I decided to re-plan the path of my PhD and therefore restated my main research question which reads as follows:
\begin{itemize}
 \item \textit{Can I use the variability of simple movements not only to automatically classify
 a human movement but also as an automatic index of users' performance using on-body sensors 
over the course of practice in the context
of Human-Robot Interaction?}
\end{itemize}
The previous question can be broken down into the following questions:
\begin{itemize}
 \item \textit{Which non-linear dynamics  techniques can provide insight in order to measure 
 the variability of simple human movements?}
 
 \item \textit{Which on-body sensors can provide reliable data 
 and which features the sensors should have in order to reliably measure human movement?}
 
 \item \textit{Which set of 
 non-linear dynamics techniques as features, 
 pre and post -processing, and machine-learning algorithms 
  can yield a reliable recognition rate to measure the human movement variability
  ?}
 
 \item \textit{Can NAO, a humanoid robot, 
 teach simple movements to a user and automatically assess user's performance 
 giving feedback to user in order to increase or decrease the variability of the movement?}
 
\end{itemize}

\subsection{Summarise of the 9 Month Report}
As stated in the 9 Month Report that was submitted in August 2015,
I am generally interested in using non-linear dynamics methods 
that can provide insight into the measurement of variability of human movements.
In particular, I started to explore the use of time-delay embedding theorem and 
Principal Component Analysis (PCA) method applied to dance activities. 
The research themes of the 9 Month Report can be summarised as follows:

\begin{itemize}
 \item Review of challenges in Human Activity Recognition using body-worn sensors.
 \item Review of non-linear tools that measure human movement variability.
 \item Review of sensing technology to capture human movement.
 \item Implementation of the Cao and mutual information algorithms in order 
    to compute the time-delay embedding parameters ($m$ and $\tau$).
 \item Implementation of a stochastic model to gain better understanding of the 
 structure of human movement concerning 
 the repeatability and trajectory of the human motion.
 \item Execution of the first pilot experiment in which data from 13 participants of different 
 levels of dance expertise were analysed with the time-delay embedding theorem and PCA methods.
\end{itemize}

For further references refer to the 9 month report \cite{mxochicale_9monthreport}.

\section{Progress}

The following section presents summaries of the PhD advances for trimesters.

\subsection{September 2015 to November 2015}

To understand the data collected from the low-cost Razor 9DOF inertial sensor
(from now on referred as razor),
I conducted further experiments with the razor's firmware in which I set 
different sensitivity values (2g,4g,8g,16g) to test the limits of the sensor
with regard to simple arm movements.
I also tested different baudrates with the ARF7044F and BlueSMiRF bluetooth dongles 
to set a sample rate for data streaming of 50 Hz

The outputs of the razor can be 3D raw or calibrated data from the accelerometer,
gyroscope and magnetometer and Euler angles.
However, as I have been studying the Euler angles, I found that both Lorenzi \textit{et al.} 
and Arsenault and Whitehead pointed out that the use of quaternion representation is more beneficial over Euler angles.
This is because the quaternion representation does not suffer from the problem of gimbal lock
and they are numerically stable 
since they do not require the calculations of many arithmetic and trigonometric operations \cite{Lorenzi2015, Arsenault2015_a, Munkundan2002}.




\subsection{December 2015 to February 2016}

I submitted a paper titled ``Dancing in Time: applying time-series analysis to Human Activity''
to the Human-Computer Interaction conference (CHI) 2016 in order to test my advances regarding the measure of dexterity 
of dance activities. For the CHI's submission I used the time-delay embedding theorem and PCA method
with the data collected from 13 participants. However, the paper was rejected and 
I have learnt from the reviewers' comments that the proposed methodology is too specific 
because it is only useful for specific axis of the sensor and with specific 
embedded values and also because it is a non-transferable proposal to other applications
because it was just tested with limited dance steps of a particular dance style.
Additionally, the data set consisting of 1 expert, 
1 intermediate and 11 novice dancers, was too small to perform statistics.
% The reviewers pointed out that research in handwriting recognition presents metrics 
% that might help us to recognise the dexterity of dance activities.

In order to improve my previous experiment,
I conducted a second pilot experiment in which seven participants danced 
six basic salsa steps with and without music.
For this experiment I included anthropomorphic data of the participants 
(gender, age, handedness, height, weight and ethnic group).
%  in order to have a better experiment and with the idea to
%  find any any extra relationship with the outcomes and the anthropomorphic data of the participants. 
However, I did not analyse the data because I found that the
sampling rate of the low-cost (Razor 9DOF) IMU sensors was different to 50 Hz \textbf{[CHECK DATA SAMPLES*]}
%  since less than 50 samples were obtained for one second. 

On the other hand, to gain a better understanding of the variability of human movement,
I follow the work of Hammerla  \textit{et al.} \cite{Hammerla2011} in order to implement a stochastic model 
that considers two normal random variables one to model the repeatability of the activity 
and the other one to model the structure of the motion.
Nonetheless, Hammerla's model considered a constant value for the periodicity of the movement 
which leaves room to develop the model further.
I therefore added a third normal random variable 
in order to explore the variation of the length in time of the repetition.
I however believe that further experiments 
are required in order to vary of the length the repetition in a human-like way.




\subsection{March 2016 to May 2016}

A third pilot experiment was performed as a validation test
for the sensors because of some issues with regard to syncronisation, sample rate and drift.
The experiment consists of six simple movements 
(static, horizontal, vertical, diagonal, circular and 8-shape)
which were performed by six participants. 
For data collection,
a low-cost Razor 9DOF
and commercial shimmer sensors were attached to the wrist of the participants.
Similarly, I created a list of commercial IMUs
(9DOF Razor, myAHRS+, EXLs3, WAX9, Xsens sensors MTw Awinda DK Lite, shimmer and Muse)
which included:
Price, Connectivity, Sensor range for accelerometer, gyroscope and magnetometer, 
sample rate, temperature, syncronisation, orientation output, battery time and API.


In order to test the advances of my research, the following body of work were submitted ``minor`` conferences:

\textbf{(i)} a Poster Abstract Submission to the University of Birmingham research poster conference
with the title ``Measuring the Variability of Human Movement,``
in which I learnt to do public engagement. I deliver my work in a friendly way to audiences of 
different background and ages.

\textbf{(ii)} an extended abstract (2 pages) and its poster submission to the The Fifth ACM International 
Symposium on Pervasive Displays with the title: 
``Understanding movement variability of simplistic gestures using an inertial sensor.``
For this work, I presented the outcome of the six participants performed six simple arm movements
in which I applied the time-delay embedding theorem, PA and percentage of cumulative energy 
to characterise variability of the movements. 
We also propose that such method can be useful
to determine different states of interactions with the display of user’s behavior 
(enthusiasm, boredom, tiredness or confusion) over the course of training, practice or rehabilitation.
According to reviewers' feedback, 
further experiments are required to show what exactly PCA lacks to yield insightful outcomes
and show more evidence about the variability within and between participants.

\textbf{(iii)} an extended abstract (2 pages) submission to the 2nd International Symposium on Wearable Robotics
with the title: ''Analysis of the Movement Variability in Dance Activities using Wearable Sensors.`` 
For this abstract,
I analysed the data from thirteen participants who repeatedly dance two salsa steps (simple and complex) for 20 seconds. 
I then applied the time-delay embedding and PCA to obtain the reconstructed state space
for visual assessment of the variability of dancers. 
However, reviewers mentioned that further explanation for the time-delay embedding theorem are required
and that neither an analysis nor a quantification were presented except for the
graphics that were linked with their level of skillfulness of the participants. 






On the other hand,
I applied to the European computational motor control summer school
in order to gain better understanding about the biomechanics of the human body 
but I was not accepted because of the high number of applications.

\subsection{June 2016 to August 2016}

Following my plans to do public engagement, I presented a poster at the XIV Symposium of Mexican Students in the U.K. 
at the University of Edinburgh in which I received a prize for one of the two best posters presented.

In order to explore classification algorithms for human activities,
I am exploring the use of The Gesture Recognition Toolkit (GRT), a machine learning library,
which contains 15 machine-learning algorithms and 16 pre-processing, post-processing, and feature-extraction algorithms 
\cite{Gillian2014}. Similarly, I proposed the use of the Georgia Tech Gesture Toolkit \cite{Westeyn2003}
which is based on the Hidden Markov Model Toolkit (HTK 3.5), I have been however 
doing little advances since I need to make sense of the data that is fed to the HTK toolkit.

On the other hand, I am doing experiments with the drift in the accelerometer and gyroscope sensors 
over short and long acquisition periods for two Razor 9DOF IMUS sensors using Robot Operating System (ROS).

\section{Publication Plan}

Two journals for publication 
were selected based on the relation with my research questions:


\begin{enumerate}
 \item Journal Submission: Human Movement Science - Elsevier [Impact factor: 1.606] (December 2016).
 
I plan to report the pros and cons of different nonlinear dynamics techniques 
(Time-delay embedding, Empirical Mode Decomposition, Lyapunov exponent, fractal dimensionality and Poincaré maps)
using data from IMUS of simple upper body movements.
My aim with this publication is to contribute to field of human movement 
with a better understanding of the use of non-linear dynamics techniques 
in order to measure the variability of simple movements. 
For further details about the experiment refer to Appendix B.


\item Journal Submission: IEEE Transactions on Pattern Analysis and Machine Intelligence.
[Impact factor: 6.077] (March 2017).

I plan to apply nonlinear dynamics techniques as a pre-processing technique 
to test different machine learning algorithms of the GTR in order 
to automatically classify the variability of human movements. 

\end{enumerate}




\section{Work Plan} 
To tackle the research questions,
six tasks(T) for the following year are planned as follow:

\begin{itemize}
\item T1 [September]: Buy 7 neMEMSi (quoted for 1070 \euro{} in August 2016) \cite{neMEMSi2016}. 
Set the sensors and the experiment  for data collection. 

\item T2 [October]: Collect data of simple movements from 12 participants in six seasons. 

\item T3 [November]: Analyse the data using non-linear dynamics techniques to gain understanding of the 
variability for within and across participants.

\item T4 [December]: Write up and submit a journal to Human Movement Science -- Elsevier.

\item T5 [January/February 17]: 
Use the data collected on October 2015 with the Gesture Recognition Toolkit 
to test different pre-processing, post-processing, feature-extraction and machine-learning algorithms.

\item T6 [March 17]: Write up and submit a journal to IEEE Transactions on Pattern Analysis and Machine Intelligence.

\textbf{\item T7 [April/May 17]: ?
\item T8 [June/July/August 17]: ? 
\item T8 [Sep/Oct/Nov 17]: ? }

\end{itemize}


\section{Conclusion}
In conclusion, I have been establishing and learning 
from the literature (refer to Appendix A). I run the second and the third experiment
because of the following reasons: 
(i) the data was too small to perform statistics; 
(ii) the sample rate were not fixed to 50 Hz;
(iii) there were problems with the synchronisation and drift of the sensors; and 
(iv) simple movements were chosen instead of complex movements.
For the data of the first experiment, I noted that the embedded values ($m$ and $\tau$) were 
only computed from the expert dancer and the same values were also used for 
the intermediate and novice dancers, which means that further tests have to be done in order to validate the 
effect the embedded values over different participants.

In terms of scientific publications two short abstracts were accepted at
(i) the Fifth ACM International Symposium on Pervasive Displays, and 
(ii) the Second International Symposium on Wearable Robotics.
From this submissions, I have learnt that further work is 
required such as the pros and cons of using the time-delay embedding theorem and PCA
and quantification of the variability.

From the technical side, I found some problems when using two or more 
low-cost inertial sensors such as the synchronisation, the drift over short and long period of acquisition.
I therefore created a list of commercial sensors in order to compare their performance and select
one sensor with reliable 3D resolution, to which I select the neMEMSi sensor
because the 3D resolution is quite similar to the MT20i sensor from Xsens.

For future plans, a four experiment is going to be performed (refer to Appendix B)
and a further experiment with a humanoid robot are going to be performed (refer to Appendix C).


\appendices

\section{Extensive, up to date literature survey}

%  Newell and Corcos stated that variability is an inherent characteristic of human movement .

The aim of automatic activity recognition is to provide information about a user's activity
generally by means of still images and video. 
However, this constrained environment using cameras has caused a shift toward the use of body-worn sensors \cite{bulling2014}.
Such sensors that are commonly used include accelerometer, gyroscope and magnetometer
for applications such as detection of falls, movement and analysis of body 
or a subject's postural orientation to mention but a few \cite{Mukhopadhyay2014}.

Although the advances in Human-Activity Recognition (HAR) have been providing good results in terms of recognition rates,
there is little research investigating the automatic identification of variability in human activity recognition.
Bulling \textit{et al.}, for instance, stated that one of the common challenges in HAR 
using body-worn sensors is \textit{intraclass variability} which occurs when 
an activity is performed differently either by a single person or several people \cite{bulling2014} . 
Lim \textit{et al.}, for example, performed an empirical study to test the motion variability 
presented between 20 gestures with 12 participants. %in which each gesture was perfomed three times.
Data was collected on hand orientation using a Microsoft Kinect sensor \cite{Lim2012}. % as reference coordinates.
As expected, due to the intrinsic variability of human movement \cite{newell1993variability}, 
there was statistical significant variability of the length of trace and speed of gesture movements.
However, Lim \textit{et al.} stated that ``the gesture type did not show significant effect of the variation`` \cite{Lim2012}.

On the other hand, another possible source of variability is the displacement of body-worn sensors. 
For instance, Haratian \textit{et al.} investigated the inadvertent changes in the position of on-body sensors 
due to rapid movements or displacement of sensors during different trials and seasons.
They proposed the use of functional-PCA 
which separates deterministic and stochastic components of movements
in order to filter and interpret, what they called, ``the true nature of movement data variability'' 
\cite{Haratian2012,Haratian2014,Haratian2016}.



Regarding the sensor brands, 
Commotti \textit{et al.} presented neMEMSi which is a microelectromechanical system (MEMS)  based on 
inertial and magnetic system-on-bard with embedding processing and wireless communication.
For validation purposed the neMEMSi was compared with the state-of-the-art device Xsens MTi-30
in which the 3D static orientation accuracy is 0.057 degrees average on Roll, Pitch and Yaw 
and 3D dynamic orientation accuracy is 0.55 degrees average on Roll, Pitch and Yaw \cite{Comotti2014}.

Furthermore, Galizzi \textit{et al.} performed power consumption tests with the 
neMEMSi-TEG for Thermo-Electric-Generators in order to increase the lifetime of the batteries.
They found that there is a trade off between accuracy, power consumption and sampling rate.
It can be said that the use of a gyroscope strongly affects the increase of power consumption
and the static and dynamic error are within 1 degree and 10 degrees respectively 
when the sampling rate is higher than 50 Hz \cite{Galizzi2015}.

neMEMSi sensors has been used for Parkinson's Disease patients' rehabilitation 
in a Timed-Up-and-Go test, where data was gathered and analysed from 
13 PD participants (mean age: 16.6$\pm$9) and 4 control subjects (mean age: 16.3$\pm$4) \cite{Caldara2014}.

Similarly, neMEMSi-Smart has been used to assess the motor performance of elderly people
in a six-minute walk test, using five adults with no pathologies (mean age: 31$\pm$6) and four elderly people with Type 2 Diabetes 
(mean age: 70.8$\pm$7) \cite{Caldara2015}.
Further experiments were conducted by Lorenzi \textit{et al.} in which the neMEMSi was attached to the head 
%,where the mass center of the sensors oscillates in the $y$ direction,
of 5 participants  with Parkinson's Disease in order to 
automatically classify those human motion disorders with an Artificial Neural Network \cite{Lorenzi2015}.
However, the neck join added signals from many postural problems
and irregular movements because of the Parkinson Disease.
Therefore, in the most recent work of Lorenzi \textit{et al.} two neMEMSi sensors were attached to the shins
of 16 patients for fine detection of gait patterns
which results in a ``good'' performance in terms of sensitivity, precision and accuracy of 
the detection of freezing of gait (FOG) for elderly people with Parkinson's Disease  \cite{Lorenzi2016}.



Arsenault and Whitehead collected data from 10 individuals performing six gestures fifty times each,
this lead to 3000 samples in total, with 500 samples per gesture. 
For recognition purposes, they reported an improvement of the classification rates
in terms of speed and accuracy using Markov Chain instead of Hidden Markov Models \cite{Arsenault2015_a, Arsenault2015_b}.
They used a network of InvenSense MPU-6050 (3-axis accelerometer and 3-axis gyroscope) sensors with the  PIC24 microcontroller.
  
% Another important point to make when you are recognising gestures is the segmentation or windowing,
% Recently, Banos \textit{et al.} 
% demonstrated that large window size does not lead good recognition performance.
% Therefore using a data set of 17 participants performing 33 fitness activities 
% reported that short windows (0.25-0.5 s) lead to better recognition performances \cite{Banos2014}. 
% 
 
\section{Detailed Description of Preliminary Experiment}

\subsection{Aim}
Apply non-linear dynamics methods to time-series from inertial sensors of 
simple movements.

\subsection{Materials and Methods}
Data collection from 12 participants was performed. 
Each participant is going to 
perform 7 simple movements (static standing, static in T position, horizontal,
vertical, diagonal, circular and eight-shape) with their arms for three minutes per movement
in six seasons.
Three sensors were attached to the wrist, forearm and upperarm of the participants.

Using the data set, it is planned to apply different non-linear techniques 
(Empirical Mode Decomposition, Lyapunov Exponent, Fractal Dimensionality, Poincar\'e Maps).
Also, the GRT is going to be used to apply techniques of 
pre-processing, post-processing and feature-extraction algorithms.

\subsection{Results and Publication}
By analysing the data using the non-linear techniques, I expect to
gain better insight to assess variability of simple movements.
I am therefore going to submit the outcomes of this experiment to
the journal Human Movement Science by Elsevier (Impact factor: 1.606) in December 2016.


\section{Further Experimenting}

In order to automatically assess the variability of simple movements, 
I am going to implement a Human-Robot Interaction application 
with NAO humanoid robot \cite{NAO}
in which simple movements will be performed by NAO 
and participants are going to replicate the movements.


% use section* for acknowledgment
\ifCLASSOPTIONcompsoc
  % The Computer Society usually uses the plural form
  \section*{Acknowledgments}
\else
  % regular IEEE prefers the singular form
  \section*{Acknowledgment}
\fi

Miguel Xochicale gratefully acknowledges the studentship from 
the National Council for Science and Technology (CONACyT) Mexico
to pursue his postgraduate studies at University of Birmingham U.K.

\ifCLASSOPTIONcaptionsoff
  \newpage
\fi



% trigger a \newpage just before the given reference
% number - used to balance the columns on the last page
% adjust value as needed - may need to be readjusted if
% the document is modified later
%\IEEEtriggeratref{8}
% The "triggered" command can be changed if desired:
%\IEEEtriggercmd{\enlargethispage{-5in}}

% references section

% can use a bibliography generated by BibTeX as a .bbl file
% BibTeX documentation can be easily obtained at:
% http://www.ctan.org/tex-archive/biblio/bibtex/contrib/doc/
% The IEEEtran BibTeX style support page is at:
% http://www.michaelshell.org/tex/ieeetran/bibtex/
%\bibliographystyle{IEEEtran}
% argument is your BibTeX string definitions and bibliography database(s)
%\bibliography{IEEEabrv,../bib/paper}
%
% <OR> manually copy in the resultant .bbl file
% set second argument of \begin to the number of references
% (used to reserve space for the reference number labels box)
% \begin{thebibliography}{1}
% 
% \bibitem{IEEEhowto:kopka}
% H.~Kopka and P.~W. Daly, \emph{A Guide to \LaTeX}, 3rd~ed.\hskip 1em plus
%   0.5em minus 0.4em\relax Harlow, England: Addison-Wesley, 1999.
% 
% \end{thebibliography}

% \nocite{*}
\bibliographystyle{IEEEtran}
\bibliography{references}


% biography section
% 
% If you have an EPS/PDF photo (graphicx package needed) extra braces are
% needed around the contents of the optional argument to biography to prevent
% the LaTeX parser from getting confused when it sees the complicated
% \includegraphics command within an optional argument. (You could create
% your own custom macro containing the \includegraphics command to make things
% simpler here.)
%\begin{IEEEbiography}[{\includegraphics[width=1in,height=1.25in,clip,keepaspectratio]{mshell}}]{Michael Shell}
% or if you just want to reserve a space for a photo:

% \begin{IEEEbiography}[{\includegraphics[width=1in,height=1.25in,clip,keepaspectratio]{mxochicale38x44.pdf}}]{name}

% \begin{IEEEbiography}{Miguel Perez-Xochicale}
% ........................
% \end{IEEEbiography}



% % if you will not have a photo at all:
% \begin{IEEEbiographynophoto}{John Doe}
% Biography text here.
% \end{IEEEbiographynophoto}
% 
% % insert where needed to balance the two columns on the last page with
% % biographies
% %\newpage
% 
% \begin{IEEEbiographynophoto}{Jane Doe}
% Biography text here.
% \end{IEEEbiographynophoto}

% You can push biographies down or up by placing
% a \vfill before or after them. The appropriate
% use of \vfill depends on what kind of text is
% on the last page and whether or not the columns
% are being equalized.

%\vfill

% Can be used to pull up biographies so that the bottom of the last one
% is flush with the other column.
%\enlargethispage{-5in}



% that's all folks
\end{document}
