%% bare_jrnl_compsoc.tex
%% V1.4a
%% 2014/09/17
%% by Michael Shell
%% See:
%% http://www.michaelshell.org/
%% for current contact information.
%%
%% This is a skeleton file demonstrating the use of IEEEtran.cls
%% (requires IEEEtran.cls version 1.8a or later) with an IEEE
%% Computer Society journal paper.
%%
%% Support sites:
%% http://www.michaelshell.org/tex/ieeetran/
%% http://www.ctan.org/tex-archive/macros/latex/contrib/IEEEtran/
%% and
%% http://www.ieee.org/

%%*************************************************************************
%% Legal Notice:
%% This code is offered as-is without any warranty either expressed or
%% implied; without even the implied warranty of MERCHANTABILITY or
%% FITNESS FOR A PARTICULAR PURPOSE! 
%% User assumes all risk.
%% In no event shall IEEE or any contributor to this code be liable for
%% any damages or losses, including, but not limited to, incidental,
%% consequential, or any other damages, resulting from the use or misuse
%% of any information contained here.
%%
%% All comments are the opinions of their respective authors and are not
%% necessarily endorsed by the IEEE.
%%
%% This work is distributed under the LaTeX Project Public License (LPPL)
%% ( http://www.latex-project.org/ ) version 1.3, and may be freely used,
%% distributed and modified. A copy of the LPPL, version 1.3, is included
%% in the base LaTeX documentation of all distributions of LaTeX released
%% 2003/12/01 or later.
%% Retain all contribution notices and credits.
%% ** Modified files should be clearly indicated as such, including  **
%% ** renaming them and changing author support contact information. **
%%
%% File list of work: IEEEtran.cls, IEEEtran_HOWTO.pdf, bare_adv.tex,
%%                    bare_conf.tex, bare_jrnl.tex, bare_conf_compsoc.tex,
%%                    bare_jrnl_compsoc.tex, bare_jrnl_transmag.tex
%%*************************************************************************


% *** Authors should verify (and, if needed, correct) their LaTeX system  ***
% *** with the testflow diagnostic prior to trusting their LaTeX platform ***
% *** with production work. IEEE's font choices and paper sizes can       ***
% *** trigger bugs that do not appear when using other class files.       ***
% The testflow support page is at:
% http://www.michaelshell.org/tex/testflow/


% \documentclass[10pt,journal,compsoc]{IEEEtran}
\documentclass[11pt,journal,onecolumn,compsoc]{IEEEtran}


 
 
%
% If IEEEtran.cls has not been installed into the LaTeX system files,
% manually specify the path to it like:
% \documentclass[10pt,journal,compsoc]{../sty/IEEEtran}



\usepackage[switch]{lineno}
\linenumbers
\leftlinenumbers
% http://tex.stackexchange.com/questions/20368/lineno-for-2-column



% Some very useful LaTeX packages include:
% (uncomment the ones you want to load)


% *** MISC UTILITY PACKAGES ***
%
%\usepackage{ifpdf}
% Heiko Oberdiek's ifpdf.sty is very useful if you need conditional
% compilation based on whether the output is pdf or dvi.
% usage:
% \ifpdf
%   % pdf code
% \else
%   % dvi code
% \fi
% The latest version of ifpdf.sty can be obtained from:
% http://www.ctan.org/tex-archive/macros/latex/contrib/oberdiek/
% Also, note that IEEEtran.cls V1.7 and later provides a builtin
% \ifCLASSINFOpdf conditional that works the same way.
% When switching from latex to pdflatex and vice-versa, the compiler may
% have to be run twice to clear warning/error messages.






% *** CITATION PACKAGES ***
%
\ifCLASSOPTIONcompsoc
  % IEEE Computer Society needs nocompress option
  % requires cite.sty v4.0 or later (November 2003)
  \usepackage[nocompress]{cite}
\else
  % normal IEEE
  \usepackage{cite}
\fi
% cite.sty was written by Donald Arseneau
% V1.6 and later of IEEEtran pre-defines the format of the cite.sty package
% \cite{} output to follow that of IEEE. Loading the cite package will
% result in citation numbers being automatically sorted and properly
% "compressed/ranged". e.g., [1], [9], [2], [7], [5], [6] without using
% cite.sty will become [1], [2], [5]--[7], [9] using cite.sty. cite.sty's
% \cite will automatically add leading space, if needed. Use cite.sty's
% noadjust option (cite.sty V3.8 and later) if you want to turn this off
% such as if a citation ever needs to be enclosed in parenthesis.
% cite.sty is already installed on most LaTeX systems. Be sure and use
% version 5.0 (2009-03-20) and later if using hyperref.sty.
% The latest version can be obtained at:
% http://www.ctan.org/tex-archive/macros/latex/contrib/cite/
% The documentation is contained in the cite.sty file itself.
%
% Note that some packages require special options to format as the Computer
% Society requires. In particular, Computer Society  papers do not use
% compressed citation ranges as is done in typical IEEE papers
% (e.g., [1]-[4]). Instead, they list every citation separately in order
% (e.g., [1], [2], [3], [4]). To get the latter we need to load the cite
% package with the nocompress option which is supported by cite.sty v4.0
% and later. Note also the use of a CLASSOPTION conditional provided by
% IEEEtran.cls V1.7 and later.


  


% *** GRAPHICS RELATED PACKAGES ***
%
\ifCLASSINFOpdf
  \usepackage[pdftex]{graphicx}
   \graphicspath{{figures/}}
  % declare the path(s) where your graphic files are
  % \graphicspath{{../pdf/}{../jpeg/}}
  % and their extensions so you won't have to specify these with
  % every instance of \includegraphics
  % \DeclareGraphicsExtensions{.pdf,.jpeg,.png}
\else
  % or other class option (dvipsone, dvipdf, if not using dvips). graphicx
  % will default to the driver specified in the system graphics.cfg if no
  % driver is specified.
  % \usepackage[dvips]{graphicx}
  % declare the path(s) where your graphic files are
  % \graphicspath{{../eps/}}
  % and their extensions so you won't have to specify these with
  % every instance of \includegraphics
  % \DeclareGraphicsExtensions{.eps}
\fi
% graphicx was written by David Carlisle and Sebastian Rahtz. It is
% required if you want graphics, photos, etc. graphicx.sty is already
% installed on most LaTeX systems. The latest version and documentation
% can be obtained at: 
% http://www.ctan.org/tex-archive/macros/latex/required/graphics/
% Another good source of documentation is "Using Imported Graphics in
% LaTeX2e" by Keith Reckdahl which can be found at:
% http://www.ctan.org/tex-archive/info/epslatex/
%
% latex, and pdflatex in dvi mode, support graphics in encapsulated
% postscript (.eps) format. pdflatex in pdf mode supports graphics
% in .pdf, .jpeg, .png and .mps (metapost) formats. Users should ensure
% that all non-photo figures use a vector format (.eps, .pdf, .mps) and
% not a bitmapped formats (.jpeg, .png). IEEE frowns on bitmapped formats
% which can result in "jaggedy"/blurry rendering of lines and letters as
% well as large increases in file sizes.
%
% You can find documentation about the pdfTeX application at:
% http://www.tug.org/applications/pdftex






% *** MATH PACKAGES ***
%
\usepackage[cmex10]{amsmath}
% A popular package from the American Mathematical Society that provides
% many useful and powerful commands for dealing with mathematics. If using
% it, be sure to load this package with the cmex10 option to ensure that
% only type 1 fonts will utilized at all point sizes. Without this option,
% it is possible that some math symbols, particularly those within
% footnotes, will be rendered in bitmap form which will result in a
% document that can not be IEEE Xplore compliant!
%
% Also, note that the amsmath package sets \interdisplaylinepenalty to 10000
% thus preventing page breaks from occurring within multiline equations. Use:
%\interdisplaylinepenalty=2500
% after loading amsmath to restore such page breaks as IEEEtran.cls normally
% does. amsmath.sty is already installed on most LaTeX systems. The latest
% version and documentation can be obtained at:
% http://www.ctan.org/tex-archive/macros/latex/required/amslatex/math/


\usepackage{amsfonts} % to use $\mathbb{Z}$


% *** SPECIALIZED LIST PACKAGES ***

\usepackage{algorithm}
\usepackage{algorithmic}
% algorithmic.sty was written by Peter Williams and Rogerio Brito.
% This package provides an algorithmic environment fo describing algorithms.
% You can use the algorithmic environment in-text or within a figure
% environment to provide for a floating algorithm. Do NOT use the algorithm
% floating environment provided by algorithm.sty (by the same authors) or
% algorithm2e.sty (by Christophe Fiorio) as IEEE does not use dedicated
% algorithm float types and packages that provide these will not provide
% correct IEEE style captions. The latest version and documentation of
% algorithmic.sty can be obtained at:
% http://www.ctan.org/tex-archive/macros/latex/contrib/algorithms/
% There is also a support site at:
% http://algorithms.berlios.de/index.html
% Also of interest may be the (relatively newer and more customizable)
% algorithmicx.sty package by Szasz Janos:
% http://www.ctan.org/tex-archive/macros/latex/contrib/algorithmicx/




% *** ALIGNMENT PACKAGES ***
%
%\usepackage{array}
% Frank Mittelbach's and David Carlisle's array.sty patches and improves
% the standard LaTeX2e array and tabular environments to provide better
% appearance and additional user controls. As the default LaTeX2e table
% generation code is lacking to the point of almost being broken with
% respect to the quality of the end results, all users are strongly
% advised to use an enhanced (at the very least that provided by array.sty)
% set of table tools. array.sty is already installed on most systems. The
% latest version and documentation can be obtained at:
% http://www.ctan.org/tex-archive/macros/latex/required/tools/


% IEEEtran contains the IEEEeqnarray family of commands that can be used to
% generate multiline equations as well as matrices, tables, etc., of high
% quality.




% *** SUBFIGURE PACKAGES ***
%\ifCLASSOPTIONcompsoc
%  \usepackage[caption=false,font=footnotesize,labelfont=sf,textfont=sf]{subfig}
%\else
%  \usepackage[caption=false,font=footnotesize]{subfig}
%\fi
% subfig.sty, written by Steven Douglas Cochran, is the modern replacement
% for subfigure.sty, the latter of which is no longer maintained and is
% incompatible with some LaTeX packages including fixltx2e. However,
% subfig.sty requires and automatically loads Axel Sommerfeldt's caption.sty
% which will override IEEEtran.cls' handling of captions and this will result
% in non-IEEE style figure/table captions. To prevent this problem, be sure
% and invoke subfig.sty's "caption=false" package option (available since
% subfig.sty version 1.3, 2005/06/28) as this is will preserve IEEEtran.cls
% handling of captions.
% Note that the Computer Society format requires a sans serif font rather
% than the serif font used in traditional IEEE formatting and thus the need
% to invoke different subfig.sty package options depending on whether
% compsoc mode has been enabled.
%
% The latest version and documentation of subfig.sty can be obtained at:
% http://www.ctan.org/tex-archive/macros/latex/contrib/subfig/




% *** FLOAT PACKAGES ***
%
%\usepackage{fixltx2e}
% fixltx2e, the successor to the earlier fix2col.sty, was written by
% Frank Mittelbach and David Carlisle. This package corrects a few problems
% in the LaTeX2e kernel, the most notable of which is that in current
% LaTeX2e releases, the ordering of single and double column floats is not
% guaranteed to be preserved. Thus, an unpatched LaTeX2e can allow a
% single column figure to be placed prior to an earlier double column
% figure. The latest version and documentation can be found at:
% http://www.ctan.org/tex-archive/macros/latex/base/


%\usepackage{stfloats}
% stfloats.sty was written by Sigitas Tolusis. This package gives LaTeX2e
% the ability to do double column floats at the bottom of the page as well
% as the top. (e.g., "\begin{figure*}[!b]" is not normally possible in
% LaTeX2e). It also provides a command:
%\fnbelowfloat
% to enable the placement of footnotes below bottom floats (the standard
% LaTeX2e kernel puts them above bottom floats). This is an invasive package
% which rewrites many portions of the LaTeX2e float routines. It may not work
% with other packages that modify the LaTeX2e float routines. The latest
% version and documentation can be obtained at:
% http://www.ctan.org/tex-archive/macros/latex/contrib/sttools/
% Do not use the stfloats baselinefloat ability as IEEE does not allow
% \baselineskip to stretch. Authors submitting work to the IEEE should note
% that IEEE rarely uses double column equations and that authors should try
% to avoid such use. Do not be tempted to use the cuted.sty or midfloat.sty
% packages (also by Sigitas Tolusis) as IEEE does not format its papers in
% such ways.
% Do not attempt to use stfloats with fixltx2e as they are incompatible.
% Instead, use Morten Hogholm'a dblfloatfix which combines the features
% of both fixltx2e and stfloats:
%
% \usepackage{dblfloatfix}
% The latest version can be found at:
% http://www.ctan.org/tex-archive/macros/latex/contrib/dblfloatfix/




%\ifCLASSOPTIONcaptionsoff
%  \usepackage[nomarkers]{endfloat}
% \let\MYoriglatexcaption\caption
% \renewcommand{\caption}[2][\relax]{\MYoriglatexcaption[#2]{#2}}
%\fi
% endfloat.sty was written by James Darrell McCauley, Jeff Goldberg and 
% Axel Sommerfeldt. This package may be useful when used in conjunction with 
% IEEEtran.cls'  captionsoff option. Some IEEE journals/societies require that
% submissions have lists of figures/tables at the end of the paper and that
% figures/tables without any captions are placed on a page by themselves at
% the end of the document. If needed, the draftcls IEEEtran class option or
% \CLASSINPUTbaselinestretch interface can be used to increase the line
% spacing as well. Be sure and use the nomarkers option of endfloat to
% prevent endfloat from "marking" where the figures would have been placed
% in the text. The two hack lines of code above are a slight modification of
% that suggested by in the endfloat docs (section 8.4.1) to ensure that
% the full captions always appear in the list of figures/tables - even if
% the user used the short optional argument of \caption[]{}.
% IEEE papers do not typically make use of \caption[]'s optional argument,
% so this should not be an issue. A similar trick can be used to disable
% captions of packages such as subfig.sty that lack options to turn off
% the subcaptions:
% For subfig.sty:
% \let\MYorigsubfloat\subfloat
% \renewcommand{\subfloat}[2][\relax]{\MYorigsubfloat[]{#2}}
% However, the above trick will not work if both optional arguments of
% the \subfloat command are used. Furthermore, there needs to be a
% description of each subfigure *somewhere* and endfloat does not add
% subfigure captions to its list of figures. Thus, the best approach is to
% avoid the use of subfigure captions (many IEEE journals avoid them anyway)
% and instead reference/explain all the subfigures within the main caption.
% The latest version of endfloat.sty and its documentation can obtained at:
% http://www.ctan.org/tex-archive/macros/latex/contrib/endfloat/
%
% The IEEEtran \ifCLASSOPTIONcaptionsoff conditional can also be used
% later in the document, say, to conditionally put the References on a 
% page by themselves.




% *** PDF, URL AND HYPERLINK PACKAGES ***
%
\usepackage{url}
% url.sty was written by Donald Arseneau. It provides better support for
% handling and breaking URLs. url.sty is already installed on most LaTeX
% systems. The latest version and documentation can be obtained at:
% http://www.ctan.org/tex-archive/macros/latex/contrib/url/
% Basically, \url{my_url_here}.





% *** Do not adjust lengths that control margins, column widths, etc. ***
% *** Do not use packages that alter fonts (such as pslatex).         ***
% There should be no need to do such things with IEEEtran.cls V1.6 and later.
% (Unless specifically asked to do so by the journal or conference you plan
% to submit to, of course. )


% correct bad hyphenation here
\hyphenation{op-tical net-works semi-conduc-tor  Birmingham}


\begin{document}
%
% paper title
% Titles are generally capitalized except for words such as a, an, and, as,
% at, but, by, for, in, nor, of, on, or, the, to and up, which are usually
% not capitalized unless they are the first or last word of the title.
% Linebreaks \\ can be used within to get better formatting as desired.
% Do not put math or special symbols in the title.
% \title{Bare Demo of IEEEtran.cls\\ for Computer Society Journals}
\title{
% Automatic Identification of Human Movement Variability
 Automatic Identification of \\ Human Movement Variability
}





%
%
% author names and IEEE memberships
% note positions of commas and nonbreaking spaces ( ~ ) LaTeX will not break
% a structure at a ~ so this keeps an author's name from being broken across
% two lines.
% use \thanks{} to gain access to the first footnote area
% a separate \thanks must be used for each paragraph as LaTeX2e's \thanks
% was not built to handle multiple paragraphs
%
%
%\IEEEcompsocitemizethanks is a special \thanks that produces the bulleted
% lists the Computer Society journals use for "first footnote" author
% affiliations. Use \IEEEcompsocthanksitem which works much like \item
% for each affiliation group. When not in compsoc mode,
% \IEEEcompsocitemizethanks becomes like \thanks and
% \IEEEcompsocthanksitem becomes a line break with idention. This
% facilitates dual compilation, although admittedly the differences in the
% desired content of \author between the different types of papers makes a
% one-size-fits-all approach a daunting prospect. For instance, compsoc 
% journal papers have the author affiliations above the "Manuscript
% received ..."  text while in non-compsoc journals this is reversed. Sigh.

\author{
% Miguel Xochicale
 	Miguel~Xochicale,~\IEEEmembership{Doctoral~Researcher;}\\
         Chris~Baber,~\IEEEmembership{Lead~Supervisor;}
         ~Martin~Russell,~\IEEEmembership{Co-Supervisor;} \\
         and~?~?,~\IEEEmembership{Academic~Advisor.}
         % <-this % stops a space

 \IEEEcompsocitemizethanks{\IEEEcompsocthanksitem 
 M. Xochicale, C. Baber and M. Russell are with the School of Electronic, Electrical and Systems Engineering, 
 The University of Birmingham, U.K. \protect\\
 % note need leading \protect in front of \\ to get a newline within \thanks as
 % \\ is fragile and will error, could use \hfil\break instead.
 E-mail: see http://mxochicale.github.io/
 }% <-this 
% stops an unwanted space
\thanks{Manuscript received August 15, 2016; revised Month Day, 2016.}
}

% note the % following the last \IEEEmembership and also \thanks - 
% these prevent an unwanted space from occurring between the last author name
% and the end of the author line. i.e., if you had this:
% 
% \author{....lastname \thanks{...} \thanks{...} }
%                     ^------------^------------^----Do not want these spaces!
%
% a space would be appended to the last name and could cause every name on that
% line to be shifted left slightly. This is one of those "LaTeX things". For
% instance, "\textbf{A} \textbf{B}" will typeset as "A B" not "AB". To get
% "AB" then you have to do: "\textbf{A}\textbf{B}"
% \thanks is no different in this regard, so shield the last } of each \thanks
% that ends a line with a % and do not let a space in before the next \thanks.
% Spaces after \IEEEmembership other than the last one are OK (and needed) as
% you are supposed to have spaces between the names. For what it is worth,
% this is a minor point as most people would not even notice if the said evil
% space somehow managed to creep in.

% The paper headers
\markboth{Twenty-oneth Month Report, August~2016}%
{Shell \MakeLowercase{\textit{et al.}}: Bare Demo of IEEEtran.cls for Computer 
Society Journals}
% The only time the second header will appear is for the odd numbered pages
% after the title page when using the twoside option.
% 
% *** Note that you probably will NOT want to include the author's ***
% *** name in the headers of peer review papers.                   ***
% You can use \ifCLASSOPTIONpeerreview for conditional compilation here if
% you desire.

% The publisher's ID mark at the bottom of the page is less important with
% Computer Society journal papers as those publications place the marks
% outside of the main text columns and, therefore, unlike regular IEEE
% journals, the available text space is not reduced by their presence.
% If you want to put a publisher's ID mark on the page you can do it like
% this:
%\IEEEpubid{0000--0000/00\$00.00~\copyright~2014 IEEE}
% or like this to get the Computer Society new two part style.
%\IEEEpubid{\makebox[\columnwidth]{\hfill 0000--0000/00/\$00.00~\copyright~2014 IEEE}%
%\hspace{\columnsep}\makebox[\columnwidth]{Published by the IEEE Computer Society\hfill}}
% Remember, if you use this you must call \IEEEpubidadjcol in the second
% column for its text to clear the IEEEpubid mark (Computer Society jorunal
% papers don't need this extra clearance.)

% use for special paper notices
%\IEEEspecialpapernotice{(Invited Paper)}

% for Computer Society papers, we must declare the abstract and index terms
% PRIOR to the title within the \IEEEtitleabstractindextext IEEEtran
% command as these need to go into the title area created by \maketitle.
% As a general rule, do not put math, special symbols or citations
% in the abstract or keywords.
\IEEEtitleabstractindextext{%


 
% \begin{abstract}
% ...
% \end{abstract}


% % Note that keywords are not normally used for peerreview papers.
% \begin{IEEEkeywords}
% X;X;X;
% \end{IEEEkeywords}



}



% make the title area
\maketitle


% To allow for easy dual compilation without having to reenter the
% abstract/keywords data, the \IEEEtitleabstractindextext text will
% not be used in maketitle, but will appear (i.e., to be "transported")
% here as \IEEEdisplaynontitleabstractindextext when the compsoc 
% or transmag modes are not selected <OR> if conference mode is selected 
% - because all conference papers position the abstract like regular
% papers do.
\IEEEdisplaynontitleabstractindextext
% \IEEEdisplaynontitleabstractindextext has no effect when using
% compsoc or transmag under a non-conference mode.



% For peer review papers, you can put extra information on the cover
% page as needed:
% \ifCLASSOPTIONpeerreview
% \begin{center} \bfseries EDICS Category: 3-BBND \end{center}
% \fi
%
% For peerreview papers, this IEEEtran command inserts a page break and
% creates the second title. It will be ignored for other modes.
\IEEEpeerreviewmaketitle





\IEEEraisesectionheading{\section{Introduction}\label{sec:introduction}}

\subsection{Aim}

This report presents the progress of the PhD research project titled
``Automatic Identification of Human Movement Variability'' 
for the period between September 2015 and August 2016.

\subsection{Background}
As stated in the 9 Month Report that was submitted in August 2015,
I am generally interested in using nonlinear dynamics methods 
that can provide insight into the variability of human activities.
Particularly, I explored the use of the time-delay embedding and PCA methods
applied to dance activities. 

The research themes of the 9 Month Report can be summarised as follows:

\begin{itemize}
 \item Revision of challenges in Human Activity Recognition using body-worn sensors.
 \item Revision of non-linear tools that measure variability.
 \item Revision of sensing technology to capture dance activities.
 \item Implementation of the Cao and mutual information algorithms in order 
    to compute the Time-delay Embedding parameters $m$ and $\tau$.
 \item Implementation of an stochastic model to gain better understanding of the 
 structure of human movement considering the repeativility and structure of the motion.
 \item Exceution of a preliminary experiment in which data from 13 participants of different 
 levels of dance expertise were analysed with the time-delay Embedding and PCA methods.
 However, data collection for this experiment were corrupted due to the impression of sampling rate.
\end{itemize}

For further references refer to the 9 Month report \textbf{[ADD REF]}.


\section{Progress}

\subsection{September 2015 to November 2015}

I did research regarding the calculations of Euler angles from the
low-cost (Razor 9DOF) IMU sensors as well as the sensor placement on the body.
Additionally, I explore PCA and its use and analyse artificial signals with added noise.




\subsection{December 2015 to February 2016}
The paper submission titled ``Dancing in Time: applying time-series analysis to Human Activity''
was rejected at the Human-Computer Interaction conference (CHI) 2016. From the reviewers' comments, 
I have learnt that the proposed methodology is too specific and it is not transferable for other applications.
Additionally, the data set consisting of 1 expert, 
1 intermeiate and 11 novice dancers, was too small to make statistically significant conclusions.
However, reviewers pointed out that research in handwriting recognition presents metrics 
that might help us to recognise the dexterity of dance activities.

Addionally, a third pilot experiment was performed in which 7 participants danced 
six basic salsa steps with and without music.
The experiment includes anthropomorphic data of the participants 
(gender, age, handedness, height, weight and ethnic group).
However, I did not analyse the data becuase I found that the
 sampling rate of the low-cost (Razor 9DOF) IMU sensors were different to 50 Hz.
%  since less than 50 samples were obtained for one second. 

To gain a better understanding of the varialibility of human movement,
I follow the work of Hammerla  \textit{et al.} \cite{Hammerla2011} in order to implement a stochastic model 
that considers the repeativility and structure of the motion.
Hammerla's model considers a constant period per repetition
which leaves room to develop the model further.
Therefore, I added a normalised random vector for frequency which basically varies the frequency 
(therefore the period) per repetition using gaussian random parameters (mean and standard deviation).

\subsection{March 2016 to May 2016}

I restated my research question which reads as follows:
\begin{itemize}
 \item \textit{Can I use the variability of simplistic movements not only to automatically identify
an activity but also as an automatic index of users' performance over the course of practice?}
\end{itemize}


A fourth pilot experiment was performed due to the problems with sensor syncronisation, sample rate and drift.
The experiment consists of six simplistic movements 
(static, horizontal, vertical, diagonal, circular and 8-shape)
which were performed by six participants. 
For data collection low-cost (Razor 9DOF) IMU sensors were attached to the wrist of the participants.

I submitted the following body of work:
(i) a Poster Abstract Submission to the University of Birmingham research poster conference.;
(ii) an extended abstract (2 pages)  and its poster submission to the The Fifth ACM International 
Symposium on Pervasive Displays; and
(iii) an extended abstract (2 pages) submission to the 2nd International Symposium on Wearable Robotics.
I also applied to the European computational motor control summer school,
however I was not accepted due to the high number of applications.

A valiation test using the Razor 9DOF sensors and shimmer sensors
was performed 
due to the fluctuation of the sample rate of the low-cost sensors.
I also performed a benchmark for commertial IMUS
(9DOF Razor, myAHRS+, EXLs3, WAX9, Xsens sensors MTw Awinda DK Lite, shimmer and Muse)
which included:
Price, Connectivity, Sensor range for accelerometer, gyroscope and magnetometer, 
sample rate, temperature, battery time and API.



\subsection{June 2016 to August 2016}

For data analysis, I proposed the use of the Georgia Tech Gesture Toolkit \cite{Westeyn2003}
which is based on the Hidden Markov Model Toolkit (HTK).
I therefore installed HTK 3.5 on a machine with Ubuntu 14.01 x64.

I also proposed to use The Gesture Recognition Toolkit (GRT) as a machine learning library.
GTR contains 15 machine-learning algorithms and 16 pre-processing, post-processing, and feature-extraction algorithms 
\cite{Gillian2014}.

I presented a poster at the XIV Symposium of Mexican Students in the U.K. at the University of Edinburgh
in which I received a prize for one of the two best posters presented.

I am testing the drift in the accelerometer and gyroscope sensors over long acquistion periods for two Razor 9DOF IMUS sensors. 
I am using Robot Operating System (ROS) to collect and process data from the sensors.
For further experiments, I am also planning  to create a Human-Robot Interaction with NAO Humanoid Robot in ROS.


\section{Publication Plan}

\begin{enumerate}
 \item Journal Submission: Human Movement Science - Elsavier [Impact factor: 1.606] (December 2016).
I plan to report the use of different nonlinear techniques 
(Empirical Mode Decomposition, Lyapunov exponent, fractal dimensionality, poincare maps)
with the pre-processing and post-processing techniques on GRT using data from IMUs of simplistic human movements.

My aim is to gain a better understanding of the use of techniques and tools,
in order to better measure the variability of simplistic activities

\item Journal Submission: IEEE Transactions on Pattern Analysis and Machine Intelligence.
[Impact factor: 6.077] (April 2017).
I plan to apply nonlinear techniques as a pre-processing technique 
to test different machine learning algorithms of the GTR in order 
to automatically clasify the variability of human movements. 


\end{enumerate}




\section{Work Plan} 
The gantt chart 1 presents a monthly breakdown for the next six months.

\section{Conclusion}


Generally, I have been establishing and learning 
from the literature (refer to Appendix A), running and planning experiments (refer to Appendix B)
and facing technical problems with the low-cost inertial sensors. 

For the previous experiments, I noted that the embedded values ($m$ and $\tau$)  were 
only computed from the expert dancer and 
the same values were also used for the intermediate and novice dancers,
which means that further tests have to be done in order to validate the 
effect the embedded values have for different participants.

In terms of scientific publications two short abstracts were accepted in
(i) the Fifth ACM International Symposium on Pervasive Displays, and 
(ii) the Second International Symposium on Wearable Robotics.

For future plans, a preliminary experiment is going to be performed (refer to Appendix B)
and further experiment with an humanoid robot is going to be performed (refer to Appendix C).


\appendices

\section{Extensive, up to date literature survey}

%  Newell and Corcos stated that variability is an inherent characteristic of human movement .

The aim of automatic activity recognition is to provide information about a user's activity
generally by means of still images and video. 
However, this constrained environment using cameras has caused a shift toward the use of body-worn sensors \cite{bulling2014}.
Such sensors that are commonly used include accelerometer, gyroscope and magnetometer
for applications such as detection falls, movement and analysis of body 
or a subject's postural orientation to mention but a few \cite{Mukhopadhyay2014}.

Although the advances in HAR has been providing good results in terms of recognition rates.
There is little research investigating the automatic identification of variability in human activity recognition.
Bulling \textit{et al.}, for instance, stated that one of the common challenges in HAR 
using body-worn sensors is \textit{intraclass variability} which occurs when 
an activity is performed differently either by a single person or several people \cite{bulling2014} . 
Lim \textit{et al.}, for example, performed an empirical study to test the motion variabilty 
presented between 20 gestures with 12 participants. %in which each gesture was perfomed three times.
Data was collected on hand orientation using a Microsoft Kinect sensor \cite{Lim2012}. % as reference coordinates.
As expected, due to the intrinsic variability of human movement \cite{newell1993variability}, 
there was statistical significant variability of the lenght of trace and speed of gesture movements.
However, Lim \textit{et al.} stated that ``the gesture type did not show significant effect of the variation`` \cite{Lim2012}.

On the other hand, another possible source of variability is the displacement of the body-worn sensors. 
For instance, Haratian \textit{et al.} investigated the inadvertent changes in the position of on-body sensors 
due to rapid movements or displacement of sensors during different trials and seasons.
They proposed the use of functional-PCA 
which separates determinist and sthocastic components of the movements
in order to filter and interpret, what they called, ``the true nature of movement data variabitliy'' 
\cite{Haratian2012,Haratian2014,Haratian2016}.



Regarding the sensor brands, 
Commotti \textit{et al.} presented neMEMSi which is a microelectromechanical systems (MEMS)  based inertial and magnetic 
system-on-bard with embedding processing and wireless communication.
For validation purposed the neMEMSi was compared with respect to the state-of-the-art device Xsense MTi-30
in which the 3D static orientation acuracy is 0.057 degrees average on Roll, Pitch and Yaw 
and 3D dynamic orientation acuracy is 0.55 degrees average on Roll, Pitch and Yaw \cite{Comotti2014}.

Furthermore, Galizzi \textit{et al.} performed power consumption tests with the 
neMEMSi-TEG for Thermo-Electric-Generators in order to increase the lifetime of the batteries.
They found that there is a trade off between accuracy, power consumtion and sampling rate.
It can be said that the use of a gyroscope strongly affects the increase of power consumption
and the static and dynamic error are within 1 degree and 10 degrees respectively 
when the sampling rate is higher than 50 Hz \cite{Galizzi2015}.

neMEMSi sensors has been used for Parkinson's Disease patients' rehabilitation 
in a Timed-Up-and-Go test, where data was gathered and analysed from 
13 PD participants (mean age: 16.6$\pm$9) and 4 control subjects (mean age: 16.3$\pm$4) \cite{Caldara2014}.

Similarly, neMEMSi-Smart has been used to assess the motor performance of elderly people
in a six-minute walk test, using five adults with no pathologies (mean age: 31$\pm$6) and four elderly people with Type 2 Diabetes 
(mean age: 70.8$\pm$7) \cite{Caldara2015}.
Further experiments were made by Lorenzi \textit{et al.} in which the neMEMSi were attached to the head 
%,where the mass center of the sensors oscillates in the $y$ direction,
of 5 participants  with Parkinson's Disease in order to 
automatically classify those human motion dissorders with an Artificial Neural Network \cite{Lorenzi2015}.
However, the neck join added signals from many postural problems
and irregular movements because of the Parkinson Disease.
Therefore, in the most recent work of Lorenzi \textit{et al.} two neMEMSi sensors were attached to the shins
of 16 patients for fine detection of gait patterns
which results in a ``good'' peformance in terms of sensitivity, precision and accuracy of 
the detection of freezing of gait (FOG) for elderly people with Parkinson's Disease  \cite{Lorenzi2016}.

Both Lorenzi \textit{et al.} and Arsenault and Whitehead 
pointed out that the use of quaternion representation is more benefical over 
other rotational representations such as the Euler angles.
For instance, quaternion representation does not suffer from the problem of gimbal lock
and they are numerically stable 
since they do not require the calculations of many trigonometric functions \cite{Lorenzi2015, Arsenault2015_a, Munkundan2002}.

Arsenault and Whitehead collected data from 10 individuals performing six gestures fifty times each,
this lead to 3000 samples in total, with 500 samples per gesutre. 
For recognition purposes, they reported an improvement of the classification rates
in terms of speed and accuracy using Markov Chain instead of Hidden Markov Models \cite{Arsenault2015_a, Arsenault2015_b}.
They used a network of InvenSense MPU-6050 (3-axis accelerometer and 3-axis gyroscope) sensors with the  PIC24 microcontroller.
  
% Another important point to make when you are recognising gestures is the segmentation or windowing,
% Recently, Banos \textit{et al.} 
% demonstrated that large window size does not lead good recognition performance.
% Therefore using a data set of 17 participants performing 33 fitness activities 
% reported that short windows (0.25-0.5 s) lead to better recognition performances \cite{Banos2014}. 
% 
% 
% 
% 
\section{Detailed Description of Preliminary Experiment}

\subsection{Aim}
Apply non-linear methods to time-series from intertial sensors of 
simplistic movements.

\subsection{Materials and Methods}


Data collection from 12 participants will be performed. 
Each participant is going to 
perform 7 simplistic movements (static standing, static in T position, horizontal,
vertical, diagonal, circular and eight-shape) with their arms for three minutes per movement
in six seasons.
Three sensors will be attached to the wrist, forearm and upperarm of the participants.

Using the data set, it is planned to apply different nonlinear techniques 
(Empirical Mode Decomposition, Lyapunov exponent, fractal dimensionality, poincare maps).
Also, the GRT is going to be used to apply techniques of 
pre-processing, post-processing and feature-extraction algorithms.

\subsection{Results and Publication}

By analysing the data using the nonlinear techniques, I expect to
gain better insight to asses variablity of simplistic movements.
I am therefore going to submit the outcomes of this experiment to
the journal Human Movement Science by Elsavier (Impact factor: 1.606) on December 2016.


\section{Further Experimenting}

In order to automatically access the variability of simplistic movements, 
I am going to implement a Human-Robot Interaction application 
with NAO humanoid robot \cite{NAO}
in which simplistic movements will be performed by NAO 
and participants are going to replicate the movements.


% use section* for acknowledgment
\ifCLASSOPTIONcompsoc
  % The Computer Society usually uses the plural form
  \section*{Acknowledgments}
\else
  % regular IEEE prefers the singular form
  \section*{Acknowledgment}
\fi

Miguel Xochicale gratefully acknowledges the studentship from 
the National Council for Science and Technology (CONACyT) Mexico
to pursue his postgraduate studies at University of Birmingham U.K.

\ifCLASSOPTIONcaptionsoff
  \newpage
\fi



% trigger a \newpage just before the given reference
% number - used to balance the columns on the last page
% adjust value as needed - may need to be readjusted if
% the document is modified later
%\IEEEtriggeratref{8}
% The "triggered" command can be changed if desired:
%\IEEEtriggercmd{\enlargethispage{-5in}}

% references section

% can use a bibliography generated by BibTeX as a .bbl file
% BibTeX documentation can be easily obtained at:
% http://www.ctan.org/tex-archive/biblio/bibtex/contrib/doc/
% The IEEEtran BibTeX style support page is at:
% http://www.michaelshell.org/tex/ieeetran/bibtex/
%\bibliographystyle{IEEEtran}
% argument is your BibTeX string definitions and bibliography database(s)
%\bibliography{IEEEabrv,../bib/paper}
%
% <OR> manually copy in the resultant .bbl file
% set second argument of \begin to the number of references
% (used to reserve space for the reference number labels box)
% \begin{thebibliography}{1}
% 
% \bibitem{IEEEhowto:kopka}
% H.~Kopka and P.~W. Daly, \emph{A Guide to \LaTeX}, 3rd~ed.\hskip 1em plus
%   0.5em minus 0.4em\relax Harlow, England: Addison-Wesley, 1999.
% 
% \end{thebibliography}

% \nocite{*}
\bibliographystyle{IEEEtran}
\bibliography{references}


% biography section
% 
% If you have an EPS/PDF photo (graphicx package needed) extra braces are
% needed around the contents of the optional argument to biography to prevent
% the LaTeX parser from getting confused when it sees the complicated
% \includegraphics command within an optional argument. (You could create
% your own custom macro containing the \includegraphics command to make things
% simpler here.)
%\begin{IEEEbiography}[{\includegraphics[width=1in,height=1.25in,clip,keepaspectratio]{mshell}}]{Michael Shell}
% or if you just want to reserve a space for a photo:

% \begin{IEEEbiography}[{\includegraphics[width=1in,height=1.25in,clip,keepaspectratio]{mxochicale38x44.pdf}}]{name}

% \begin{IEEEbiography}{Miguel Perez-Xochicale}
% ........................
% \end{IEEEbiography}



% % if you will not have a photo at all:
% \begin{IEEEbiographynophoto}{John Doe}
% Biography text here.
% \end{IEEEbiographynophoto}
% 
% % insert where needed to balance the two columns on the last page with
% % biographies
% %\newpage
% 
% \begin{IEEEbiographynophoto}{Jane Doe}
% Biography text here.
% \end{IEEEbiographynophoto}

% You can push biographies down or up by placing
% a \vfill before or after them. The appropriate
% use of \vfill depends on what kind of text is
% on the last page and whether or not the columns
% are being equalized.

%\vfill

% Can be used to pull up biographies so that the bottom of the last one
% is flush with the other column.
%\enlargethispage{-5in}



% that's all folks
\end{document}
