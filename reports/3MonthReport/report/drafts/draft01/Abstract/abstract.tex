%*****************************************************************************************
%*********************************** Abstract ********************************************
%*****************************************************************************************

\begin{abstract}


Human Activity Recognition (HAR) has been a challenging task, since human activity is complex 
and highly diverse. Besides, the approaches for HAR
differ practically as much as the types of activities that have been recognized 
and types of sensor data that have been used, to which we hypothesise that concepts 
from Chaos theory could be a powerful tool to tackle such inconveniences. Henceforth, 
the aim of this research proposal is to gain fully understanding of concepts from 
Chaos theory that can be used for HAR so as to provide a 
robust approach for real-time applications using two inertial sensors: 
one wrist-worn and one ankle-worn. Finding appropriate methods for characterization 
and classification of human body actions is not only motivated by the fact that the 
motion capture system should be non-intrusive and easy-to-worn 
but also that theoretical approach should be well suited for real-time applications.


\end{abstract}
