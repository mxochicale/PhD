%*****************************************************************************************
%*********************************** First Chapter ***************************************
%*****************************************************************************************

\chapter{Introduction}  %Title of the First Chapter

\ifpdf
    \graphicspath{{Chapter1/Figs/Raster/}{Chapter1/Figs/PDF/}{Chapter1/Figs/}}
\else
    \graphicspath{{Chapter1/Figs/Vector/}{Chapter1/Figs/}}
\fi

Human Activity Recognition (HAR) has been a challenging task \cite{Aggarwal2004}, 
since human body activity is complex and highly diverse \cite{Kim2010}.
HAR has many potential applications; for instance, these can be seen in personal 
assistants, surveillance, patient monitoring, sports analysis, dance activities, 
human-robot interaction and biometrics to mention but a few \cite{Aggarwal2011}. 
HAR has been provided several frameworks in recognizing primitive activities 
such as walking, jogging, cycling, jumping; nonetheless, few work has been been
done in identifying complex activities that for example involve dance. 

Dance activities reflects the dancer's profile i.e., rhythmic sense, 
home country \cite{Iwai2011}, personality, biological features (i.e., genre and age),
dancing skills (i.e., fluency of the motion, adding erractical or additional movements,
coordination and turbulence in dance, steadynees of the rhythm, predictability of 
the motion) \cite{GrammerK.ElisabethOberzaucher2011}.
Henceforth, for the current PhD work
we are focusing on the multiattribute classification of dancing activities 
since these are complex and  highly dinamical activities to identify.

On the other hand, HAR deals with many issues which are
a) the different types of activities to recongnise,
b) the selection of motion capture system which should be unobtrusive and inexpensive,
c) the selection of algorithms for feature extraction and classification,
d) and the responding time (offline or online) \cite{Lara2013}.
Yet, the chosen approach varies almost as greatly as the types of activities 
that have been recognized and types of sensor data that have been used 
\cite{Kim2010}. Additionally, HAR has got many challenges that motive our work 
to find new techniques in order to recognize activities in a more realistic conditions. 
Therefore, finding appropriate methods for HAR is not only motivated by the 
fact that the motion capture system should be non-intrusive and easy-to-worn 
but also that theoretical approach should be well suited for real-time applications.

% %%MOTION CAPTURE SYSTEMS
% There is extensive literature for motion capture systems for HAR, namely:
% vision-based 
% \cite{Forsyth2005,Reddy2012,Wang2006,Kim2011,Duric2002,Oikonomidis2012}; 
% floor-sensor based \cite{Paradiso1997, Steinhage2008, Aguilar2007,
% Wimmer2011,Yin2003,Moere2004,Richardson2004,
% Srinivasan2005,Rangarajan2008,Visell2010, Rajalingham2010};
% and intertial and foot-sensor based
% \cite{Razak2012, Bamberg2008,Benocci2009,Xu2012,Holleczek2010}.
% Of these, the latter being the least intrusive and sensors are easy-to-worn for dancing
% activities.

% %%SIGNAL CHARACTERIZATION
% The signal characterization in HAR considers time-domain (mean, standard deviation, 
% variance, interqualite range, mean absolute deviation, correlation between axes, 
% entropy and kurtosis), frequency-domain (Fourier Transform and 
% Discrete Cosine Transform),
% others (Principal Component Analysis, Linear Discriminant Analysis
% and Recurrent Qualitative Analysis \cite{GrammerK.ElisabethOberzaucher2011} ) 
% for feature extraction.
% Most recently, concepts from the nonlinear dynamics such as
% time-delay embedding \cite{J.FrankS.Mannor2010, Sama2013, Ali2007, Basharat2009},
%  Lyaponov exponents \cite{Ali2007}, cylindrical phase space, 
%  H\'enon maps and Poincar\'e maps \cite{Suzuki2013}
% have been shown to posses powerful features in activity recogntion tasks, 
% especially given the need for small memory and processing requirements of
% the current devices.



% We hypothesise that concepts from chaos theory
% could be a powerful tool to tackle such inconveniences.


Thus, the aim of the PhD is focused on a fully understanding 
of the concepts from non-linear dynamics that can be used 
as a feature extraction for machine learning algorithms so as to provide 
a robust HAR approach for real-time applications using inertial sensors.
% 
% To fulfill the previous-stated aim, the following research questions will be addressed:
% \begin{enumerate}
%   \setlength{\itemsep}{0pt}
%   \setlength{\parsep}{0pt}
%   \item 
% % Most recently, concepts from the nonlinear dynamics such as
% % time-delay embedding \cite{J.FrankS.Mannor2010, Sama2013, Ali2007, Basharat2009},
% % Lyaponov exponents \cite{Ali2007}, cylindrical phase space, 
% % H\'enon maps and Poincar\'e maps \cite{Suzuki2013}
% % have been shown to posses powerful features in activity recogntion tasks, 
%   
%   Which non-reported concepts from non-linear dynamics could be use to obtain 
%   features in the human body analysis?
% %   \item Passive dynamic walking model has been used to investigate nonlinear gait 
% % 	dynamic \cite{Kurz2007}. 
% % 	Does the development of models for diffent human body activities 
% % 	would be an accurate approach for real-time recognition?
% %   \item Would be suitable to build models based on the reconstructed attractors 
% % 	\cite{Nikulchev2013} for different human body activities in real-time 
% % 	applications?
% %   \item It has been reported that up to 50 human activities can be recongnized using 
% % 	video based approached \cite{Reddy2012}.
% % 	How many human body activities and with what accuracy can be recognized
% % 	using the current proposal?
%   \item How can motion for the wrist, the ankle and the hip be quantified so as 
%         to set features for the better HAR? 
% %   \item Where is the optimal place to wear the sensor for 
% % 	the recognition of dancing activities?
%   \item Which axis(es) among accelerations, gyroscope and magnetometer 
% 	will provide the best information for different HAR?
%   \item Machine learning
% \end{enumerate}


% Given this questions, 
This three month report is organised as follows: First,
the state-of-the-art in motion capture systems, machine learning approaches in HAR
and human body analysis using nonlinear dynamics are reviewed in Section 2. Sencond, 
the propose framework and the workplan is shown is presented in section 3.



