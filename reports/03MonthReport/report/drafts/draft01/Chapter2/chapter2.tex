%*****************************************************************************************
%*********************************** Second Chapter **************************************
%*****************************************************************************************

\chapter{Previous Work}

\ifpdf
    \graphicspath{{Chapter2/Figs/Raster/}{Chapter2/Figs/PDF/}{Chapter2/Figs/}}
\else
    \graphicspath{{Chapter2/Figs/Vector/}{Chapter2/Figs/}}
\fi

The central goal of the research proposal is focused on a full understanding of 
concepts from nonlinear dynamics that can be used for human activity recognition 
so as to provide a robust approach for real-time identification using inertial sensors.
Thus, reviews of motion capture systems are presented. Additionally, 
different approaches for HAR 
that use concepts from nonlinear dynamics are reviewed. 

\section{Motion Capture Systems}
Motion capture systems can be chategorised into three approaches: 
vision-based \cite{Forsyth2005}; 
floor-sensor based
\cite{Paradiso1997,Steinhage2008,Aguilar2007,Wimmer2011,Yin2003,Moere2004,
Richardson2004,Srinivasan2005, Rangarajan2008,Visell2010, Rajalingham2010}
; and inertial-sensor based
\cite{Razak2012,Bamberg2008,Benocci2009,Xu2012,Holleczek2010}.
Although vision-based and floor-sensor based are rooted in non-intrusive 
motion capture systems, these are still required to be used into the space 
where users are constrained to move around.
On the other hand, wearable systems have been proven to be the least instrusive 
and easy-to-use sensors. However, the choosen approach for human activity 
recognition varies as greately as the 
types of activities that have been recognized 
and many other factors such as type of sensors, data connection protocol, obstrusiveness, 
recognition performance, energy consumption, flexibility, computational processing,
features, learning, and accuracy 
are considered for the performance of the motion capture system \cite{Lara2013}.

% The first approach has been obtained good results when the main task is to render 
% humans with very realistic static appearance.
% % , and this is accomplished when 
% % users wear either passive markers (tight-fitting black clothing with small white spots) 
% % or active markers (flashing infrared lights attached to the body) in an open space 
% % covered with a collection of cameras within which people move around.
% Nonetheless, three are the main drawbacks of these approach. First, high temporal 
% resolution is required both in time and in space to deal with sharp accelerations 
% caused by some kinds of motion (hitting, jumping, etc.).
% The second issue comes when one should determine the reported marker position 
% in which the frame corresponds to a particular location on the body. The third difficulty 
% is that the markers need to be visible to at least two cameras with sufficient baseline,
% and the correspondence needs to be unambiguous \cite{Forsyth2005}.

% The second approach, floor-sensor based, is especially suitable for non-intrusive 
% human body interaction which overcome some of the previous-mentioned drawbacks.
% In \cite{Paradiso1997}, for instance, Paradiso \textit{et al.} developed a sensor 
% arrangement hidden under a 6x10 foot carpet to monitor dynamic foot position and 
% pressure.
% A capacitive sensor mat was designed to monitor movement trajectories  by
% \cite{Steinhage2008}, also Aguilar \textit{et al.} \cite{Aguilar2007} created a 
% low-cost capacitive human-detection systems which allow to detect the position of 
% the person and direction of motion. In similar fashion, Wimmer \textit{et al.} 
% \cite{Wimmer2011} proposed a system of capacitive sensors for whole body interaction. 
% XSensor pressure pad is used for creating an animation interface in \cite{Yin2003}. 
% Infosense \cite{Moere2004}, a studio for designing interactive experiences in sensate 
% space, use pressure sensitive mats for capturing users' activity and motion.
% Richardson \textit{et al.} \cite{Richardson2004} developed a prototype of modular nodes,
% which join together to form a flexible, pixellated, pressure-sensing surface.
% Srinivasan \textit{et al.} \cite{Srinivasan2005} designed a high-resolution sensing floor 
% that provides real-time data about pressure and relative location as well as orientation 
% and movement of the users. Similarly, Rangarajan \textit{et al.} \cite{Rangarajan2008} 
% proposed a large sensing area and the floor system has been integrated with a marker 
% based motion capture system. 
% 
% However, few techniques have been developed for capturing 
% touch via the feet over a distributed display as \cite{Visell2010, Rajalingham2010}. 
% Although, previous approaches are rooted in non-intrusive, 
% these are still subjected to be used into the space where users are constrained to move
% around.

% The third approach, foot-sensor based, could also be considered as in-shoe systems,
% which are essentially mobile and wearable systems for monitoring activities of daily life
% and offered a better efficiency, flexibility, mobility and also reduced cost
% \cite{Razak2012}.  For instance, Bamberg \textit{et al.} \cite{Bamberg2008} build a 
% GaitShoe which is worn in any shoe to collect data unobtrusively, in any environment, 
% over long time periods. A wireless wearable systems was designed by \cite{Benocci2009}, 
% which is able to monitor both inertial and pressure information from the feet.
% Xu \textit{et al.} \cite{Xu2012} designed the Smart Insole which takes advantages 
% of the ubiquitous Internet access so as to monitor the user gait in real-time and over 
% long periods. On the other hand, Holleczek \textit{et al.} \cite{Holleczek2010} developed 
% a textile pressure sensor which is particularly comfortable to wear for snowboarding 
% activities. Nonetheless, one of the limitation of these approaches
% is that when using Inertial-sensor, systems should be in calibration routines prior 
% to its use. Additiationally, the spatial resolution of the data is low compared to 
% floor-sensor based systems due to fewer sensors.

% \section{Wearable Systems in HAR}

% Although wearable systems have been proven to be the least instrusive and easy-to-use 
% sensors, the choosen approach for human activity recognition varies as greately as the 
% types of activities that have been recognized 
% % (Table.~\ref{t:typesofHAR})
% and many other factors such as type of sensors, data connection protocol, obstrusiveness, 
% recognition performance, energy consumption, flexibility, computational processing,
% features, learning, and accuraty  \cite{Lara2013} .
% (Table.~\ref{t:typesofONLINEHAR} and 
% Table.~\ref{t:typesofOFFLINEHAR}) have to be considered.

% In addition to that, Lara \textit{et. al.} \cite{Lara2013} presented various 
% open problems where no previous work has been done and that could be addressed as 
% applications of this research proposal:
% 1) Obtain models to classify individuals with similar characteristics
% such as age, weight, gender, health condition so as to have a more effective 
% recognition model for overweight young model, one for normal male children,
% another one for elderly female, among others; and
% 2) Collect human activity data in order to estimate levels of sedentarism, 
% exercise habits or health conditions in a target population.

\section{Machine Learning Aproaches in HAR}
Much attention has also been given in recent years to use machine learning 
algorithms in HAR since the identification of activities entail a large number of 
attribute values and different transition points between activities;
to this end several approaches have been used i.e. 
Suppor Vector Machines \cite{J.FrankS.Mannor2010, Sama2013, Schuldt2004},
template matching \cite{Nguyen2009,Lin2007}, 
Hidden Markov Model 
\cite{Kohn2012,Niu2004,Chen2003,Bernardin2003,Eickeler1998,Chang2000},
Dynamic Time Warping \cite{Bautista2013,Boulgouris2004,Celebi2011},
Neural Networks \cite{Rosenblum1994,Ji2013,Modi2011,Boesnach2004},
and  most recently Dynamic Bayesian Networks \cite{Cuaya2013, Wang2014}, 
Emerging Patterns \cite{Tao2009, Kim2010},
Conditional Random Field  \cite{Wang2006} and
Skip Change Conditional Random Field \cite{Kim2010}. 
However, much research remains to be done to find suitable approaches 
in identifying activities in more realistic conditions.


% \begin{table}[h]
% \centering
% \caption{Types of activities recognized by Human Activity Recognition
% Systems \cite{Lara2013}
% }
% \label{t:typesofHAR}
% \scriptsize{
% \begin{tabular}{|c|l |}
% \hline
% \textbf{Group }& \textbf{Activities} \\ \hline
% Ambulation (AMB)& Walking, running, sitting, standing still, liying, climbing stairs \\ 
%            & descending stairs, riding escalator, and riding elevator. \\ \hline
% Transportation (TR)& Riding a bus, cycling and driving. \\ \hline
% Daily activities (DA)& Eating, drinking, working at the PC, watching TV,\\ 
%               & reading, brushing teeth, stretching, scrubbing, and vacuumming. \\ \hline
% Exercise  (EX)& Rowing, lifting weights, spinning, Nordic walking \\ 
%        & and doing push ups. \\ \hline
% Military (MIL)& Crawling, kneeling, situation assessment and  \\ 
%           & openning a door. \\ \hline
% Upper body (UB)& Chewing, speaking, swallwing, sighing, and moving the head\\ \hline  
% Others (OT)& Dancing different styles of music: latin, waltz, salsa, etc. \\ \hline  
% \end{tabular}}
% \end{table}

% 
%  \begin{table}[h]
%  \centering
% \caption{List of Abbreviations and Acronyms \cite{Lara2013}}
% \label{t:ABBR}
% \tiny{
% \begin{tabular}
%  { c | l }
%  \toprule
%  \textbf{ACC} & Accelerometers \\ 
%  \textbf{ANN} & Artificial Neural Network \\
%  \textbf{ALR} & Additive Logistic Regression clasifier \\
%  \textbf{AR} & Auto-Regressive model coefficients \\
%  \textbf{C4.5} & C4.5 decision tree classifier \\
%  \textbf{CART} & Classification And Regression Tree \\
%  \textbf{DCT} & Discrete Cosine Transform \\
%  \textbf{DTW} & Dynamic Time Warping \\
%  \textbf{DT} & Decision Tree-based classifier \\
%  \textbf{ENV} & Environmental sensors \\
%  \textbf{FBF} & Fuzzy Basis Function \\
%  \textbf{FD} & Frequency-domain features \\
%  \textbf{GYR} & Gyroscope \\
%  \textbf{HMM} & Hidden Markov Model\\
%  \textbf{HW} & House Activities \\
%  \textbf{KNN} & \emph{k-}Nearest Neighbors classifier \\
%  \textbf{LAB} & Laboratory controlled experiment\\
%  \textbf{LDA} & Linear Discriminant Analysis\\
%  \textbf{LS} & Least Squares algorithm \\
%  \textbf{MNL} & Monolithic classifier \\
%  \textbf{NAT} & Naturalistic experiment\\
%  \textbf{NB} & The Na\"{\i}ve Bayes classifier \\
%  \textbf{N/S} & Not Specified\\
%  \textbf{PCA} & Principal Component Analysis\\
%  \textbf{PHO} & Activities related to phone usage\\
%  \textbf{PR} & Polinomical Regression\\
%  \textbf{SMCRF} & Semi-Markovian Conditional Random Field\\
%  \textbf{SPC} & User-specific classifier \\
%  \textbf{TA} & Tilt Angle\\
%  \textbf{TD} & Time-domain features\\
%  \textbf{TF} & Transient Features\\
%  \textbf{VS} & Vital signals\\
%  \bottomrule
%  \end{tabular}}
%  \end{table}
% 
% 
% \begin{table}[h]
% \centering
% \caption{Summary of Online HAR Systems \cite{Lara2013}
% }
% \label{t:typesofONLINEHAR}
% \tiny{
% \begin{tabular}
% { l  >{\centering}m{1.2cm} >{\centering}m{2cm} c c c c c c  >{\centering}m{1.2cm} c  c }
% \toprule
% 
% \textbf{Reference} & \textbf{Activities} & \textbf{Sensors} & \textbf{ID} 
% & \textbf{Obstru-} & \textbf{Expe-} & \textbf{Energy} & \textbf{Flexi-} 
% & \textbf{Process-} & \textbf{Features} & \textbf{Learning} & \textbf{Accuracy}\\
%  &  &  &  & \textbf{sive} &  \textbf{riment} &  
%  & \textbf{bility} & \textbf{ing} &  &  &  \% \\
% \midrule
% 
% \rowcolor{black!10} 
% ActiServ \cite{Berchtold2010a}, \cite{Berchtold2010b} & AMB, PHO (11) & ACC (phone) & Phone & Low & N/S &  
% Low & SPC & High & $\bar{y},\sigma_y^2$ & RFIS & 71 \% - 98 \%\\
% 
% Kao \cite{Kao2009} & AMB, DA (7) & ACC (wrist) & Custom & Low & N/S & 
% Medium & MNL & Low & TD, LDA & FBF & 94.71\% \\ 
% 
% \rowcolor{black!10}
% Ermes \cite{Ermes2008} & AMB(5) & ACC (wrist, ankle, chest) & PDA & High & N/S &
% High & SPC & High & TD, FD & DT & 94\% \\
% 
% eWatch \cite{Maurer2006} & AMB(6) & ACC,ENV(wrist) & Custom & Low & LAB & 
% Low & MNL & Low & TD,FD & C4.5, NB & 94\% \\
% 
% \rowcolor{black!10}
% COSAR  \cite{Riboni2011} & AMB, DA (10) & ACC(watch, phone), GPS & Phone & Low & NAT & 
% Medium & MNL & Medium & TD & COSAR & 93\% \\ 
% 
% Vigilante \cite{Lara2012} & AMB(3) & ACC and VS (chest) & Phone & Medium & NAT &  
% Medium & MNL & Low & TD, FD, PR, TF & C4.5 & 92.6\% \\
% 
% \rowcolor{black!10}
% Tapia \cite{Tapia2007} & EXR(30) & ACC (5 places), HRM & Laptop & High & LAB& 
% High & Both & High & TD, FD, HB & C4.5, NB & 86\% (SD), 56 \% (SI)\\ 
% 
% Brezmes \cite{Brezmes2009} & AMB(5) & ACC (phone) & Phone & Low & N/S &  
% Low & SPC & High & TD, FD & KNN &  80 \% \\
% 
% % \rowcolor{blue!10}
% % &  &  &  &  &  &  
% % &  &  &  &  &  \% \\
% % 
% % &  &  &  &  &  &  
% % &  &  &  &  &  \% \\
% 
% \bottomrule
% \end{tabular}}
% \end{table}
% 
% \begin{table}[h] Chaos Theory
% \centering
% \caption{Summary of Offline HAR Systems \cite{Lara2013}
% }
% \label{t:typesofOFFLINEHAR}
% \tiny{
% \begin{tabular}
% { l c  >{\centering}m{2cm} c c c c >{\centering}m{1.5cm} >{\centering}m{1.5cm} c }
% \toprule
% 
% \textbf{Reference} & \textbf{Activities} & \textbf{Sensors} & \textbf{ID}  & \textbf{Experiment} 
% & \textbf{Obstrusive} & \textbf{Flexibility} & \textbf{Features} & \textbf{Learning} & \textbf{Accuracy}\\
% \midrule
% 
% 
% \rowcolor{black!10}
% Altun \cite{Altun2010} & AMB(19) & ACC, GYR (chest, arms, legs) & High & None & NAT & 
% MNL & PCA, SFFS & BN, LS, KNN, DTW, ANN &   87\% - 99 \% \\
% 
% Khan \cite{Khan2010} & AMB, TR (15) & ACC (chest) & Medium & Computer & NAT &  
% MNL & AR, SMA, TA, LDA & ANN &  97.9\% \\
% 
% \rowcolor{black!10}
% Pham \cite{Pham2008} & AMB, DA (4) & ACC(jacket) & Medium & N/S & N/S &  
% Both & Relative Energy & NB, HMM &  97 \% (SD), 95 \% (SI)\\
% 
% He \cite{He2009} & AMB(4) & ACC(trousers pocket) & Low & PC & N/S &  
% MNL & DCT, PCA &  SVM &   97.51\% \\
% 
% \rowcolor{black!10}
% Centinela \cite{Lara2012} &  AMB (5) & ACC and VS (chest) & Medium & Cellphone & NAT &  
% MNL & TD, FD, PR, TF & ALR, Bagging, C4.5, NB, BN & 95.7 \% \\
% 
% Jotaba \cite{Jatoba2008} & AMB (6) & ACC, SPI & High & Tablet & LAB &  
% Both&  TD / FD & CART, KNN & 86 \% (SI), 95 \%  (SD) \\
% 
% \rowcolor{black!10}
% Bao \cite{Bao2004} & AMB, DA (20) & ACC(wrist, ankle, thigh, elbow, hip) & High & None & NAT &  
% MNL & TD, FD & KNN, C4.5, NB & 94 \% \\
% 
% Hanai \cite{Hanai2009} & AMB(5) & ACC(chest) & Low & Laptop & N/S &  
% MNL & HAAR filters & C4.5 &  93.91 \% \\
% 
% \rowcolor{black!10}
% Chen \cite{Chen2008} & AMB, DA,HW & ACC (2 wrists) & Medium & N/S & LAB  &  
% MNL & TD, FD & FBF & 93 \% \\
% 
% He \cite{He2008} & AMB (4) & ACC & Low & PC & N/S &  
% MNL & AR & SVM &  92.25 \% \\
% 
% \rowcolor{black!10}
% Minnen \cite{Minnen2007} & AMB, MIL (14) & ACC (6 places) & High & Laptop & Both &  
% SPC & TD, FD & Boosting & 90 \% \\
% 
% Zhu \cite{Zhu2009} & AMB, TR (12) & ACC(wrist, waist) & High & PC & N/S &  
% SPC & AV,3DD & HMM &  90 \% \\
% 
% \rowcolor{black!10}
% Vinh \cite{Vinh2011} & AMB, DA (21) & ACC (wrist, hip) & Medium & N/S & N/S &  
% N/S & TD & SMCRF & 88.38 \% \\
% 
% Parkka \cite{Parkka2006} & AMB, DA (9) & ACC, ENV, VS (22 signals) & High & PC & NAT &  
% MNL & TD, FD & DR, KNN &  86 \% \\
% 
% \rowcolor{black!10}
% McGlynn \cite{Mcglynn2011} & DA(5) & ACC(thigh, hip, wrist) & Low & None & N/S &  
% SPC & DTW & DTW ensemble & 84.3 \% \\
% 
% Cheng \cite{Cheng2010} & UB(11) & Electrodes (neck, chest, leg, wrist) & High & PC & LAB &  
% MNL & TD & LDA & 77  \% \\
% 
% \bottomrule
% \end{tabular}}
% \end{table}



\section{Human Body Analysis Using Nonlinear Dynamics}
% In this section two approaches that use concepts from nonlinear dynamics are reviewed, 
% namely: intertial-based and video-based. It is also reviewed several cases of clinical 
% applications that have been proven to be worthwhile for the research proposal.

Recently, the use of intertial-based motion capture system in human body activity 
and gait recognition has been proposed the use of concepts from nonlinear dynamics 
that implements methods to obtain; for instance, the state space reconstruction,
determinism test, Lyapunov exponents and Poincar\'e maps.
These concepts have been proven to be efficient approaches to meet the computitional 
requirements for processing information in real time
\cite{J.FrankS.Mannor2010, Sama2013, Gouwanda2012,Perc2005, Akiduki2013,Akiduki2014}. 

Similarly, video-based approaches have been proven to present good results 
to recognize more complex human body activities such as the dextery of tennis players 
using attractors and fractal properties \cite{Yamamoto2000,Suzuki2013} 
, identification of dancing ballet, jumping, running, sitting and walking activities
using the attractors of the reconstructed state space, 
multivariate phase space reconstruction
and Maximal Lyapunov Exponent 
\cite{Ali2007,Basharat2009,Venkataraman2013}, and the recongnition of 
two-dimensional single-stroke patterns of 26 letters 
through modeling the attractor behaviors \cite{Ijspeert2013}. 

On the other hand, concepts from nonlinear dynamics also have been used to understand 
the behavior of human body activities for clinical applications, for instance
Vieten \emph{et al.} \cite{Vieten2013} quantify differences between gait patterns 
under constraints by approximating the time series data that underlying limit cycle 
attractors. 
Harbourne \emph{et al.} \cite{Harbourne2009} presented evidence 
to make diferentiation between health and nonhealth subjects and 
to identify difference between young and old people 
by analysing the changes in the attractor in the state space.
Zhang \textit{et al.} \cite{Zhang2010} demonstrated that the points of the 
Poincar\'e section are highly susceptible to noise; however, the use of power 
spectrum density analysis of the correlation coefficient demonstrate a relationship 
between $1/f$ noise and healthy subjects is very strong.
Buzzi \textit{et al.} \cite{Buzzi2003} demonstrated satisfactorily that elderly 
subjects increased the inability to compensate the natural stride-to-stride variations
by using the Lyapnov Exponent (LyE) and surrogate LyE (s-LyE).
Terrier \textit{et al.} \cite{Terrier2013} analysed and characterized the 
synchronization of steps with an auditory stimulus to evaluate gait stability 
and fall risk by using the maximum LyE.
It is important to mention that researchers in this area have been made 
a greater emphasis on the need for embedded software to make accessible 
tools for physical therapist.


% Frank \textit{et al.} \cite{J.FrankS.Mannor2010}, for instance, showed that by 
% using time-delay embedding theorem \cite{takens1981} to extract features from 
% data collected, the average accuracy for the experiments on the 
% entire data set is $85.48 \pm 0.30\%$, hence, they can obtain 100\% test set accuracy,
% they mean by using test set that every test set was associated with the correct user,
% on a noisy data set consisting of 25 individuals. 
% Data was collected using the built-in mobile phone accelerometer.
% Although, time-delay embedding approach is well explained, little is reported 
% regarding the determination of proper values to reconstruct the state space
% (the embedding dimension $m$ and the embedding delay $\tau$). Also,
% a post-processing stage were made in which data were trimmed to remove the first and 
% last few steps, and no other activities than gait recognizition were tested with their 
% approach.

% Similarly, Sam\'a  \textit{et al.} \cite{Sama2013} researched on human gait recognition
% where the prototype device was located on the waist so as to collect 
% accelerometer-based sensor data, then data is treaten as a time series
% in order to reconstruct the state space. Thier proposed approached showed that the 
% identification task can be performed by taking into account only a reduced number of 
% features from the reconstructed state space compared to Frank \textit{et al.} 
% \cite{J.FrankS.Mannor2010}.
% In this approach Singular Spectrum Analysis is applied to obtain the features of the
% reconstructed attractor. 
% The proposed approach shows an overall accuracy of 95\% 
% which was obtained by using Support Vector Machine with Gaussian kernel as classifier.
% Through extensive experiments, their approach confirmed that 
% there is no universal method to choose $\tau$, though, no conclusion was given regarding 
% the value of $m$. It is also important to note that their proposal show that optimal 
% values for  $\tau$ and $m$ do not provide the best gait recognition performance. 

% On the other hand, Perc \cite{Perc2005} provides explanations, aimed to undergraduate 
% level students, that analyse the dynamics of human gait.
% Particularly, methods to obtain the best state space reconstruction,
% determinism test and Lyapunov exponents ($LyE$) were explained in detail.
% Despite the efforts of Perc to provide a software to compute $\tau$, $m$, determinism test, 
% and  $LyE$; restrictions for the values are given, for example, $m$ should be 
% less or equal than 10, and parameters should be apropriately established to 
% replicate the same results. Besides, no other than MIT Gait Database data were analysed.
% 
% Recently, Akiduki \emph{et~al.} \cite{Akiduki2013} defined a vector field 
% \cite{Okada2003} in the same state space in order to find a matrix of linear maps from 
% the state space into a symbol space to analyse skills in human movements.
% Such approach is different from previos-mentioned research.
% Similarly, Akiduki \emph{et~al.} \cite{Akiduki2014} proposed a method to extract motion 
% knowledge from sensor data. The features of each motion are expressed as the shape 
% of the attractors in the state space, then the information of the dynamics is converted 
% in terms of symbolization. Simulations of a simple unforced pendulum demonstrate the 
% relationship between the physical condition of motion and the distribution of symbols 
% in the symbol space. Clustering method was used to evaluate the percentage of 
% recognition. It is worthwhile to mention that sensor data from accelerometer and 
% gyroscopes are considered for their approach, however, no details are given regarding 
% the expimental results in the human body activities.

% Gouwanda \textit{et al.} \cite{Gouwanda2012} demonstrated that angular rate of
% lower body extremities can be valid to extimate the maximum Lyapunov exponent ($\lambda^*$)
% which was computed using the reconstructed state space. 
% Then experiments of 10 subjects walking on a treadmill demonstrate that lower values 
% of $\lambda^*$ exhibit more dynamical stability  
% whereas systems that are less dynamical stable exhibit higher 
% values of $\lambda^*$.
% Although the motion capture system is wireless, it runs in a
% parallel proccessing workstation with LabVIEW 8.5 so as to stream data in real-time.
% The use of \textit{four} gyroscopes can be more intrusive than the wearable attaching 
% reflective markers, also sensors should be placed where minimum skin and muscle 
% movement exists.


% \subsection{Video-based Approaches}
% Video-based approaches have been proven to present good results to recognize more 
% complex human body activities such as the dextery of tennis players 
% \cite{Yamamoto2000,Suzuki2013}, dancing ballet, jumping, running, sitting and walking 
% \cite{Ali2007,Basharat2009,Venkataraman2013}, and the recongnition of 
% two-dimensional single-stroke patterns of 26 letters \cite{Ijspeert2013}. 
% 

% For example, Ali \emph{et~al.} \cite{Ali2007} 
% proposed the use of chaotic invariants for identification purposes
% such as emdedding of the data, existence of deterministic data 
% structure, compute dynamical, topological and metric invariants of the periodic orbits 
% and the use of the invariants. 
% Henceforth, two experiments were made: first, using 5 markers to capture 3D motion 
% sequences in which their approach recognized 5 activities with a mean accuracy of 89.7 \%. 
% In the second experiment 81 videos with 9 different accions were analysed
% in which six landmarks on the human body were extracted as the joint tracks
% to which a 92.6 \% of mean accuracy was obtained. For classification purposes, 
% both experiments use the K-nearest neighbor classifier. Nevertheless, 
% in the first experiment, misclassified issues were presented mainly because of the 
% similarity between run and walk and dance and walk. Besides, in the second experiment,
% most of the errors in the classification were observed in bending and jumping actions.

% In the same fashion, Basharat \emph{et~al.} \cite{Basharat2009} 
% take advantage of the regular structure in the reconstructed state space for making 
% predictions along the strange attractor using kernel regression. Since, 
% one of the primary aims of the proposal was to make predictions for human action 
% synthesis, they demonstrate that multivariate phase space reconstruction 
% is a better approach than the univariate one because the former case show normal 
% body poses. Video-based motion capture data acquired the time series of four markers
% in activities such as: walking, running, jumping and ballet.
% In addition, their approach was applied to dynamic texture synthesis where
% multivariate predictions craete more realistic and smoother videos.

% Venkataraman \emph{et~al.} \cite{Venkataraman2013} proposed a framework in which the 
% shape of the reconstructed state space can be seen as a feature for classification.
% The computation of the largest Lyapunov exponent needs large amount of data to 
% produce a reliable result, consequently, the largest Lyapunov exponent is not appropriate
% to analyse human activity where the signal observation is small and variable.
% To this end, the shape distribution approach \cite{Osada2002} was found to be stable 
% for the previous setback. The proposed method presented an accuracy of 88.6\%, 
% improving largest Lyapunov exponent (60.2\%), to clasify impaired (stroke survivors) 
% and unimpaired (neurologically normal) subjects.
% Data was collected by wearing a single marked-based on the wrist with a
% 3D motion capture system where five actions (Dance, Jump, Run, Sit, Walk) 
% were recognized with an accurary of 96.84\% that achieved the results of 
% Ali \emph{et~al.} \cite{Ali2007}.
% Their approach also was tested with a Kinect sensor with 20 actions markers that were 
% divided into 3 Action Sets so as to produce an accuracy of 79\% for 10 subjects. 
% Yet, errors were made in classification of Jump, Run and Walk which are due to the 
% similarity between the actions. Regarding the parameters for the reconstructed state space,
% a constant embedding dimension of 3 was chosen for all activities, expept
% a value of either 3 or 4 was selected on stroke rehabilidation database.
% They also stated that a good selection of $\tau$ ensure that the data are
% maximally spread that results in a smooth reconstructed space. However,
% $\tau$ was computed using first zero-crossing of autocorrelation function
% \cite{S05} in which the data should be strongly periodic.

% Yamamoto \emph{et al.} \cite{Yamamoto2000, Suzuki2013} quantified human dextery 
% of hitting movements of expert and novice tennis players where the repeated movement 
% was treated as an attractor and the swithing movements between forehead and backhand 
% strokes were treaten as transition of attractors. Poincar\'e maps and fractal 
% dimension were calculated so as to show that novice has higher fractal dimension 
% than that of experts, this suggest that dextery can be quantified in terms of 
% fractal dimension. Data was collected using the direct linear transformation 
% method \cite{aa1971} for 3D space reconstruction from two 2D images \cite{Nigg1994}. 
% The kinematics of the striking action were collected with the help of four markers 
% placed on the shoulder and hip joints of the participants. 
% Although Pincar\'e maps have been proven to be suitable to charaterize subjects' 
% dextery, there are litte detials regarding the computation of parameters for the 
% attractor reconstruction.

% On the other hand, Ijspeert \emph{et~al.} \cite{Ijspeert2013}, for instance,
% proposed to transform well-understood simple attractor 
% with the help of a learnable forcing function term into a desired attractor system.
% They also demonstrated that is possible to recongnise human motion without any specific 
% sytem tuning or sophisticated classification algorithm.
% Such result was obtained in fitting trajectories of the 26 letters which 
% were drawn by a human user into a two-dimensional single-stroke patterns.
% Hence, their approach acomplished a 87\% of recognition rate, however,
%  errors ocurred where similar-looking letter were made (e.g., Q being recognized as a G).
% 

% \subsection{Clinical Approaches}

% Concepts from Chaos theory also have been used to understand the behavior of human
% body activities. 




% 
% Vieten \emph{et~al.} \cite{Vieten2013}, for instance, successfully quantify 
% differences between gait patterns under constraints that include normal walking, 
% walking while performing a mental task (counting backwards by threes) and walking 
% with weights added to the ankles. Such results were obtained by approximated 
% the time series data that underlying limit cycle attractors, from which three 
% measurements were calculated: $\delta M$ amounts to the difference between 
% two attractors, % (a measure for the difference of two movements),
% $\delta D$ computes the difference between the two,
% associated deviations of the state vector away from the attractor
% % (a measure for the change in movement),
% and $\delta F$ a combination of the previous two.
% Attractor analysis was carried out via the software StatFree Version 7.0.3.1,
% to which no further details were given. In addition to that, two pre-processing 
% tasks were made: low-pass filtering and residual analysis to find optimal cutoff 
% frequency. Walking experiments were performed on a treadmill, where the subjects
% wore two intertial sensors mounted on the lateral aspect of each ankle with the 
% data logger, and analysis was limited to use acceleration data, however,
% their motion capture systems permits the measurement of tri-axial gyroscopes. 

% In \cite{Harbourne2009} Harbourne \emph{et~al.} provide a good review of clinical
% applications that use concepts from Chaos theory such as detrended fluctuation 
% analysis, correlation dimension, mutual information, Hurst exponent,
% symbolic entropy and recurrence quantification analysis. 
% For example, the variability of movement is inherent within all biological systems 
% and by analysing the transition of variability (the changes in the 
% attractor in the state space), one can understand the source of behavioral changes.
% Similarly, by analysing the Lyapunov exponent, conclusion were made 
% in the complexity of walking patterns, therefore the examination of 
% the \textit{LyE} show evidence to make diferentiation between health and nonhealth subjects. 
% Also, using the approximate entropy, one can determine the regularity of an activity 
% which is useful to identify difference between young and old people.
% It is important to mention that, they have been made a greater emphasis on the need 
% for embedded software to make accessible tools for physical therapist.
% There are little explanations regarding the mathematical approach
% and the motion capture systems.

% Buzzi \textit{et al.} \cite{Buzzi2003} using LyE and surrogate LyE (s-LyE) analysis
% demonstrated that stride-to-stride variations are not randomly derived and may be 
% deterministic in nature. It is important to note that they conclude that LyE increased 
% from the ankle toward the hip becuase of the ground restriction at the lower
% end and the decrease in the available degrees of freedom. Experiments on 20 subjects
% walking on a treadmill wearing 3 markers (hip, knee, and ankle) demonstrated 
% satisfactorily that elderly subjects increased the inability to compensate 
% the natural stride-to-stride variations.
% However, a relative small sample size of subjects is used in tPoincar\'e sectionhis study and 
% a larger number of them is desirable to be more conclusive regarding the
% measuraments of variability.

% Terrier \textit{et al.} \cite{Terrier2013} analysed and characterized the 
% synchronization of steps with an auditory stimulus using long-term and short-term 
% divergence exponents, which were obtained with the measurement of the maximum LyE.
% It is important to mention that short-term divergence exponent 
% is obtained in one stride (or one step) after the initial perturbation, 
% and long-term divergence takes a time interval betwen 4 and 10 strides.
% Henceforth, short-term divergence exponent is more appropriate to evaluate
% gait stability and fall risk. Also, the effect of auditory stimulus 
% on short-term divergence exponents was stronger at slow speeds
% which its related with patients with neurological disorders that tend to walk slowly. 
% However, this approach is subjected to local test in which users wear foot-pressure 
% sensors on a treadmill and the data normalize the number of strides for proper 
% interpretation of the signal.

% Zhang \textit{et al.} \cite{Zhang2010} proposed the use of Laplacian Eigenmaps 
% \cite{Belkin2002} to project the high-dimensional data into a low dimension.
% Also, it is worthwhile to note that their approach based on the calculation of the  
% correlation coefficient $C(i)$, that measure the distance between cycles 
% \cite{Zhang2006}, do not depend on a good reconstruction of the state space,  
% where the methods to choose an appropriate the embedding dimension and time lag are 
% dependant on the application at hand. Each cycle in the data is simplified into a 
% single point on the Poincar\'e section, however, the points of the Poincar\'e section
% are highly susceptible to noise that is commonly in biological data.
% To this end, the method of Zhang \textit{et al.} \cite{Zhang2006} has been proven 
% to be very 
% effective to deal with such incoveniences. The power spectrum density analysis of the 
% $C(i)$ demonstrate that a relationship between $1/f$ noise and healthy subjects is very 
% strong. Experiments were made in 10 subjects who were walking during 10 continuous 
% minutes. Subjects worn three electrogoniometers, devices for measuring joint angles,
% on the hip, knee, and ankle joints of the right leg.

