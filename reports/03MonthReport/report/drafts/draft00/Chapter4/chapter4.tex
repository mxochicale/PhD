%*****************************************************************************************
%*********************************** Fourth Chapter **************************************
%*****************************************************************************************

\chapter{Timeline}

% **************************** Define Graphics Path **************************
\ifpdf
    \graphicspath{{Chapter4/Figs/Raster/}{Chapter4/Figs/PDF/}{Chapter4/Figs/}}
\else
    \graphicspath{{Chapter4/Figs/Vector/}{Chapter4/Figs/}}
\fi

Based on the proposed framework, tasks can be planned as follow:
\begin{itemize}[noitemsep,topsep=0pt,parsep=0pt,partopsep=0pt]
%  \item T1: 5 curses will be taken and they will be suggested by the doctoral committee.
 \item T2: State-of-the-art review in human activity recognition using 
 nonlinear dynamics.
% \item T3: Definition of the experiments and motion capture system.
% \item T4: Programming method to obtain the reconstructed attractor
% \item T5: Evaluate the nonlinear measurements and the motion characterization.
\item T6: Review of state-of-the-art of machine learning methods 
for human activity recognition using wearable sensors.
\item T7: Definition of the experiments and application(s).
\item T8: Selecting the subjects to test the application(s) in order to have 
data for publications.
\item T9: Comparison of the proposed approach with recent methods.
\item T10 Writing, review and PhD thesis defence.
\end{itemize}

Timeline is shown in the gantt chart (Figure \ref{fig:ganttchart})
in which each year is divided into four periods (1 to 4).
Additionally, it is an objective to write and publish at least two articles per year.


\begin{figure}[htbp!] 
\begin{center}
  \begin{gantt}{12}{12} %{rows}{columns}
    \begin{ganttitle}
      \numtitle{2014}{1}{2014}{1}
      \numtitle{2015}{1}{2016}{4}
      \numtitle{2017}{1}{2017}{3}
    \end{ganttitle}
    
    \begin{ganttitle}
      \numtitle{1}{1}{2}{1} % Titles with numbers
      \numtitle{1}{1}{2}{1}
      \numtitle{1}{1}{4}{1}
      \numtitle{1}{1}{3}{1}
%       \numtitle{1}{1}{3}{1}
    \end{ganttitle}
    
  \ganttbar{T1}{0}{4}
  \ganttbarcon{T2}{4}{2}
  \ganttbarcon{T3}{5}{1}
  \ganttbarcon{T4}{6}{2}
  \ganttbarcon{T5}{6}{2}
  \ganttbarcon{T6}{8}{1}
  \ganttbarcon{T7}{8}{1}
  \ganttbarcon{T8}{9}{1}
  \ganttbarcon{T9}{9}{2}
  \ganttbar{T10}{10}{2}
  \end{gantt}
  \caption[PA]{Gantt Chart for the research proposal}
\label{fig:ganttchart}
  \end{center}
\end{figure}
  
  
  
