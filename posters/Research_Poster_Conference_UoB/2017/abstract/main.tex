\documentclass[a4paper]{article}



\usepackage{fancyhdr}
\pagestyle{fancy}
\fancyhf{}
\rhead{}
\lhead{Research Poster Conference UoB 2017}
\renewcommand{\headrulewidth}{0pt}





%% Language and font encodings
\usepackage[english]{babel}
\usepackage[utf8x]{inputenc}

%% Sets page size and margins
\usepackage[a4paper,top=3cm,bottom=2cm,left=3cm,right=3cm,marginparwidth=1.75cm]{geometry}


\usepackage{lineno}
\linenumbers




%% Useful packages
\usepackage{amsmath}
\usepackage{graphicx}
\usepackage[colorinlistoftodos]{todonotes}
\usepackage[colorlinks=true, allcolors=blue]{hyperref}

% \usepackage{mathdots} % to use \iddots for an inverse-diagonal dots

% \title{Can a Robot tell you how well you move?}
% \title{How well you move?: Can a Robot asnwer that?}
%\title{How Well you Move? Using Humanoid Robots to instruct you how to move.}
%\title{Do you know how well you Move? \\  Towards the use of Humanoid Robots to analyse your human movement}
% \title{How Well you Move? \\  Measuring Movement Variability in the context of \\ Human-Robot Interaction}
\title{How Well you Move? \\  Movement Variability in the context of \\ Human-Robot Interaction}
\author{Miguel P. Xochicale, Chris Baber and  Mourad Oussalah}
\date{Deadline: 10th of April 2017 \\ \today }

\begin{document}
\maketitle
\thispagestyle{fancy}


\section*{Summary of research [1500 words]}



%%%%%%%%%%%%%%%%%%%%%%%%%%%%%%%%%%%%%%%%%%%%%%%%%%%%%%%%%%%%%%%%%%%%%%%%%%%%%%%
%%%%%%%%%%%%%                 FIRST VERSION
%
% %One line of reasearch to undersand human movement is the
% %analysis of movement variability.
%
% %Movement
% Variability is an inherent feature in human movement that is presented
% both individually and between humans.
% % For instance, it has been investigated that a person who has been
% % performing certain movements for years is considered an expert
% % which means that the variability is decreasing as the person
% % is consistent and repetitive with the movement performance.
% However, little research has been done in the context
% of Human-Robot Interaction in order to quantify such variability.
% %that can be related to say how well one manage to mirror the movements of an expert.
% For instance, Guneysu et al. 2015 \cite{guneysu2015children} has been analised data from four physiotherapist
% whom were asked to perform the same upper arm movements. As it was expected,
% % to observe
% variability were presented in each repetition of the movememnts as well as
% in the movements between physiotherapits.
% %Althought the therapists have been repated the same movements for many years,
% %they tend to be variable in each repetition as well as variability
% %between therapists.
% % <!-- . Which means that we generally rely on
% % the comments and feedback from experts to improve the way we move. -->
% % Research in biomechanics has been done in establishing metrics of performances
% % that are linked with the variability of the performance of movements.
% % It seems that having a humanoid robot is of great help
% % to encourage kids, elderly people to put their bodies to move.
% %
% %However, little research has been done in the area of Human-Robot Interaction
% %
% % the human resources are sometime limited and mainly this cost money.
% % <!-- of well performed movement. -->
% Therefore, for this work we present preliminary results of an
% scenerio where it can be observed the phenomenon of human movement variability
% in the context of Human-Robot Interaction.
% The experiment consist of a Human-Robot Imitation activity
% %a front-to-front activity between a human an a humanoid robot.
% % imitation activity
% in which a humanoid robot is used as an instructor to perform consistent and repetitive movements.
% Then 20 participants were asked to mirror robot's arm in vertical and horizontal
% for normal and faster speed conditions.
% We found that our metrics based on nonlinear dynamics have the remarkable capability
% of measuring how well a paticipant imitate or mirror a humanoid robot.
% %In our work, we are developing
% %The novelty of my research relies on the creation of
% %  What we are trying to do is to create metrics
% % to measure
% % how well
% % a subject is imitating simple arm movements.
% % From this experiments,
% % we have found that even with simple movements
% % s, we have found that even with simple momvemetns
% % each persons tends to move differently.
% %The proposed metric can be useful when there is few experts instructors
% %are availble for evaluation of the movement activity.:
% With that, we view that there are many potentional applications of this research which
%  relies in fields of rehabilidation, therapy, and enterteinment to mention but a few.
% %where is little research about the measurement of the variability of movement imitation.
% %This line of reseach has a lot of potencial in rehabilitation scenecarios
% %where repetitive humans should be performed.
%


% =======

% We have been evolving with no further feedback than being adapting ourselves
% in different task that we perform to our daily life.
% The novelty of my research is
% how well a person move
% and to do that
% we have to start by evaluation simple movements of which we choose arm movements
% the way to do that is by using a humanoid which its own limitation
% can perform a consistent movement as an expert show where little varition of
% movevemtns can be seen.



%%%%%%%%%%%%%%%%%%%%%%%%%%%%%%%%%%%%%%%%%%%%%%%%%%%%%%%%%%%%%%%%%%%%%%%%%%%%%%%
%%%%%%%%%%%%%                 SECOND VERSION
%
% Variability is an inherent feature in human movement that is presented
% both individually and between humans.
% However, little research has been done in the context
% of Human-Robot Interaction in order to quantify such variability.
% For instance, Guneysu et al. 2015 \cite{guneysu2015children} has been analysed data from four physiotherapist
% whom were asked to perform the same upper arm movements. As it was expected,
% variability were presented in each repetition of the movements as well as
% in the movements between physiotherapist.
% Therefore, for this work we present preliminary results of an
% scenario where it can be observed the phenomenon of human movement variability
% in the context of Human-Robot Interaction.
% The experiment consist of a Human-Robot Imitation activity
% in which a humanoid robot is used as an instructor to perform consistent and repetitive movements.
% Then 20 participants were asked to mirror robot's arm in vertical and horizontal
% for normal and faster speed conditions.
% We found that our metrics based on nonlinear dynamics have the remarkable capability
% of measuring how well a participant imitate or mirror a humanoid robot.
% With that, we view that there are many potential applications of this research which
% relies in fields of rehabilitation, therapy, and entertainment to mention but a few.

% %%%%%%%%%%%%%%%%%%%%%%%%%%%%%%%%%%%%%%%%%%%%%%%%%%%%%%%%%%%%%%%%%%%%%%%%%%%%%%%
% %%%%%%%%%%%%%                 THIRD VERSION
% %
% Variability is an inherent feature in human movement that is presented
% both individually and between humans.
% However, little research has been done in the context
% of Human-Robot Interaction in order to quantify such variability.
% For instance, Guneysu et al. (2015) \cite{guneysu2015children} have analysed
% data from four physiotherapists who were asked to perform the same upper arm movements.
% As was expected, variability was presented in each repetition of the movements
% as well as in the movements between physiotherapists.
% Therefore, for this work we present preliminary results of a scenario where the
% phenomenon of human movement variability in the context of Human-Robot Interaction
% can be observed.
% The experiment consists of a Human-Robot Imitation activity in which a humanoid
% robot is used as an instructor to perform consistent and repetitive movements.
% Following this, 20 participants were asked to mirror the robot's arm in
% vertical and horizontal corresponding to normal and faster speed conditions.
% We found that our metrics based on nonlinear dynamics have the remarkable capability
% of measuring how well a participant imitates a humanoid robot.
% With this in mind, we believe that the potential applications of this research
% are many, for instance in the fields of rehabilitation, therapy, and entertainment
% to mention but a few.
%


% %%%%%%%%%%%%%%%%%%%%%%%%%%%%%%%%%%%%%%%%%%%%%%%%%%%%%%%%%%%%%%%%%%%%%%%%%%%%%%%
% %%%%%%%%%%%%%                 FOUR VERSION
% Variability is an inherent feature in human movement that is presented
% both individually and between humans.
% However, little research has been done in the context
% of Human-Robot Interaction in order to quantify such variability.
% For instance, Guneysu et al. (2015) \cite{guneysu2015children} have analysed
% data from four physiotherapists who were asked to perform the same upper arm movements.
% As was expected, variability was presented in each repetition of the movements
% as well as in the movements between physiotherapists.
% Therefore, for this work we present preliminary results
% %of a scenario
% where the phenomenon of human movement variability
% in the context of Human-Robot Interaction can be observed.
% The experiment consists of a Human-Robot Imitation activity in which a humanoid
% robot is used as an instructor to perform consistent and repetitive movements.
% Following this, 20 participants were asked to mirror the robot's arm in
% vertical and horizontal corresponding to normal and faster speed conditions.
% We found that our metrics based on nonlinear dynamics have the remarkable
% capability of measuring how well a participant imitates a humanoid robot.
% With this in mind, we believe that the potential applications of this research
% are many, for instance in the fields of rehabilitation, therapy,
% and entertainment to mention but a few.

% %%%%%%%%%%%%%%%%%%%%%%%%%%%%%%%%%%%%%%%%%%%%%%%%%%%%%%%%%%%%%%%%%%%%%%%%%%%%%%%
% %%%%%%%%%%%%%                 FIFTH VERSION
% Variability is an inherent feature in human movement that is presented
% both individually and between humans.
% However, little research has been done
% when quantifying such variability in the context of Human-Robot Interaction.
% For instance, Guneysu et al. (2015) \cite{guneysu2015children}
% , in an scenario of rehabilitation with children, have analysed data from
% four physiotherapists who were asked to perform the same upper arm movements.
% As was expected, the movement variability was presented in each repetition
% of the movements as well as in the movements between physiotherapists.
% Therefore, for this work we present preliminary results where the phenomenon
% of human movement variability in the context of Human-Robot Interaction
% can be observed.
% The experiment consists of a Human-Robot Imitation activity in which a humanoid
% robot is used as an instructor to perform consistent and repetitive movements.
% Following this, 20 participants were asked to mirror the robot's arm in
% vertical and horizontal corresponding to normal and faster speed conditions.
% We found that our metrics based on nonlinear dynamics have the remarkable
% capability of measuring how well a participant imitates a humanoid robot.
% With this in mind, we believe that the potential applications of this research
% are many, for instance in the fields of rehabilitation, therapy,
% and entertainment to mention but a few.




% %%%%%%%%%%%%%%%%%%%%%%%%%%%%%%%%%%%%%%%%%%%%%%%%%%%%%%%%%%%%%%%%%%%%%%%%%%%%%%%
% %%%%%%%%%%%%%                 SIXTH VERSION
% % (Amendments of Mourad Oussalah) 6 April 2017, 07:14
% Variability is an inherent feature in a human movement that occurs both at
% individual and group level. Given the complexity and at some extent the
% subjectivity that constraint the investigation of human(s) movement variability,
% a promising research direction is the study of human-robot interaction from
% the motion imitation perspective where little research has been performed so far.
% %is presented both individually and between humans.
% % However, little research has been done
% % when quantifying such variability in the context of Human-Robot Interaction.
% For instance, Guneysu et al. (2015) \cite{guneysu2015children},
% in an scenario of rehabilitation with children, have analysed data from
% four physiotherapists who were asked to perform the same upper arm movements.
% As was expected, the movement variability was presented in each repetition
% of the movements as well as in the movements between physiotherapists.
% % Therefore, for this work we present
% In the same spirit, this work presents some preliminary results
% of a Human-Robot Interaction scenario where the phenomenon
% of human movement variability
% %in the context of Human-Robot Interaction
% can be observed.
% The experiment consists of a Human-Robot Imitation activity in which a humanoid
% robot is used as an instructor to perform consistent and repetitive movements.
% A (randomly chosen) group of
% % Following this,
% 20 participants were asked to mirror the robot's arm in vertical and horizontal
% direction in normal to fast speed conditions.
%  % corresponding to normal and faster speed conditions.
% We found that our metrics based on nonlinear dynamics have
% %the remarkable
% good
% capability in measuring how well a given participant imitates the humanoid robot.
% With this in mind, we believe that the potential applications of this research
% are many, for instance in the fields of rehabilitation, therapy,
% and entertainment to mention only a few.



%%%%%%%%%%%%%%%%%%%%%%%%%%%%%%%%%%%%%%%%%%%%%%%%%%%%%%%%%%%%%%%%%%%%%%%%%%%%%%%
%%%%%%%%%%%%%                 SEVENTH VERSION
% (Amendments of Mourad Oussalah) 6 April 2017, 07:14
Variability is an inherent feature in a human movement that occurs both at
individual and group level. Given the complexity and at some extent the
subjectivity that constraint the investigation of human(s) movement variability,
a promising research direction is the study of human-robot interaction from
the motion imitation perspective where little research has been performed so far.
%is presented both individually and between humans.
% However, little research has been done
% when quantifying such variability in the context of Human-Robot Interaction.
For instance, Guneysu et al. (2015) \cite{guneysu2015children},
in an scenario of rehabilitation with children, have analysed data from
four physiotherapists who were asked to perform the same upper arm movements.
As was expected, the movement variability was presented in each repetition
of the movements as well as in the movements between physiotherapists.
% Therefore, for this work we present
In the same spirit, this work presents some preliminary results
of a Human-Robot Interaction scenario where the phenomenon
of human movement variability
%in the context of Human-Robot Interaction
can be observed.
The experiment consists of a Human-Robot Imitation activity in which a humanoid
robot is used as an instructor to perform consistent and repetitive movements.
Following this, a (randomly chosen) group of
20 participants were asked to mirror the robot's arm in vertical and horizontal
direction in normal to fast speed conditions.
 % corresponding to normal and faster speed conditions.
We found that our metrics based on nonlinear dynamics have
%the remarkable
good
capability in measuring how well a given participant imitates the humanoid robot.
With this in mind, we believe that the potential applications of this research
are many, for instance in the fields of rehabilitation, therapy,
and entertainment to mention only a few


%%%%%
%%%%% REFERENCES
%%%%%


%\bibliographystyle{alpha}
%\bibliographystyle{apalike}
%\bibliographystyle{acm}
\bibliographystyle{ieeetr}
\bibliography{references}


% \section{Experiments}


\end{document}
