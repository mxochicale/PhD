\documentclass[12pt]{article}
\usepackage[a4paper, total={6in, 10in}]{geometry}


\author{Miguel P Xochicale\\
% \author{Miguel P Xochicale \qquad Chris Baber \\
map479@bham.ac.uk \\
Department of Electronic Engineering\\
School of Engineering\\
University of Birmingham, UK}
\title{
eMOTION: Analysis of Emotion and Movement Variability in the Context of Human-Robot Interaction
} 
\date{\today}


% SHOULD BE 150 words
\begin{document}
\maketitle

Movement variability is an inherent feature within and between persons.
Such movement variability can be seen in many activities where human 
body movement is involved. For instance,
 in the case of accessing the quality of movement 
when one person has an injury in a body extremity there is 
little understanding about how well a person is recovering 
to his/her original movements.
With that in mind, we hypothesise that not only the subtle 
variations of face emotions but also simple body movements can be 
described and quantified in using nonlinear dynamics.
Nonlinear dynamics is an area of mathematics that helps to 
understand phenomena where many changing variables are involved 
such as modelling weather forecast, modelling neural activity or, in our case,
 understanding and measuring movement variability.
For this work, we therefore explain how the state space reconstruction theorem 
works (where dynamics of an unknown system can be reconstructed using one dimensional 
time series) and
present preliminary results of the use of the state reconstruction
to understand the relationship between the variability of arm movements, head
pose estimation and the emotion of six participants in the context of 
human-robot interactions.
The results of the state space reconstruction in the context of face emotions 
lead us to conclude that not only the variability of upper body movement 
can be analysed and quantified using the state space reconstruction theorem
but also the subtle variability of face emotion transitions across time 
(e.g. from excitement to neutral to boredom, etc)
can be understood and measured using nonlinear dynamics.
We also present some areas of applications of our results such rehabilitation, 
neuroscience, sport science, education, artificial intelligence
to mention but a few.

303words

\end{document}
