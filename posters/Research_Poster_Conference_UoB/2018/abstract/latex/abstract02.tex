\documentclass[12pt]{article}
\usepackage[a4paper, total={6in, 10in}]{geometry}


\author{Miguel P Xochicale\\
% \author{Miguel P Xochicale \qquad Chris Baber \\
map479@bham.ac.uk \\
Department of Electronic Engineering\\
School of Engineering\\
University of Birmingham, UK}
\title{
eMOTION: Analysis of Emotion and Movement Variability in the Context of Human-Robot Interaction
} 
\date{\today}


% SHOULD BE 150 words
\begin{document}
\maketitle

Movement variability (MV) is an inherent feature within and between persons.
Such MV can be seen in many activities where human body movement is involved. 
So, we hypothesise that not only the subtle 
variations of face expressions but also simple body movements 
in the context of human-robot interaction activities
can be described and quantified 
using nonlinear dynamics (ND) theorems.

We therefore explain how the uniform time-delay embedding theorem, a technique of ND, works 
and present both results of our hypothesis and the potential impact 
in areas such as 
rehabilitation, neuroscience or artificial intelligence to name a few.

\end{document}
