%  
%  Research Poster Conference 
%  University of Birmingham
%
% Important Dates: 
%
% All press releases applications must be received by 16.00 on the 20 March 2015.
% Posters must be submitted by 16.00 on the 4 May 2015 
%
%
% Miguel Perez-Xochicale
% <perez.xochicale@gmail.com>
% https://github.com/mxochicale
%

\documentclass{article}

\begin{document}

\title{Chaos Theory Approach to \\ Human Activity Recognition}
\author{P\'erez-Xochicale Miguel Angel }
\date{Research Poster Conference 2015 \\  Deadline: 20th March, 2015 }

\maketitle

\section*{Press Release}

The use of concepts from Chaos Theory to Human Activity Recognition (HAR) is a very innovative 
approach to tackle problems in different domains:
human behavior modeling, human-robot interaction, eHealth to mention but a few. 
This research will therefore extend the field of HAR in three significant ways.
First, by applying concepts from Chaos Theory 
(Takens' embedding theorem, Lyapunov exponent, Poincaré map)
the research will create novel analysis
and interpretation of the data. Second, by focussing on the complex human activities
involved in dancing, cycling or juggling the research will address challenges in time-series 
analysis which are beyond comtemporary research.
Third, by classifying data from the Chaos Theory analysis the investigation will 
produce realistic HAR.

This research is funded by the Mexican National Council for Science and Technology (CONACyT) 
and it is developed in the School of Eletronic, Eletric and Computer Engineering at
the University of Birmingham.


\textbf{-- 146 words 981 characters}


\end{document}
